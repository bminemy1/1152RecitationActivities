\documentclass[]{ximera}
%handout:  for handout version with no solutions or instructor notes
%handout,instructornotes:  for instructor version with just problems and notes, no solutions
%noinstructornotes:  shows only problem and solutions

%% handout
%% space
%% newpage
%% numbers
%% nooutcomes

%I added the commands here so that I would't have to keep looking them up
%\newcommand{\RR}{\mathbb R}
%\renewcommand{\d}{\,d}
%\newcommand{\dd}[2][]{\frac{d #1}{d #2}}
%\renewcommand{\l}{\ell}
%\newcommand{\ddx}{\frac{d}{dx}}
%\everymath{\displaystyle}
%\newcommand{\dfn}{\textbf}
%\newcommand{\eval}[1]{\bigg[ #1 \bigg]}

%\begin{image}
%\includegraphics[trim= 170 420 250 180]{Figure1.pdf}
%\end{image}

%add a ``.'' below when used in a specific directory.
\input{../preamble.tex} %% we can turn off input when making a master document

\title{Trigonometric substitutions}  

\begin{document}
\begin{abstract}		\end{abstract}
\maketitle



\begin{comment}
\section{Warm up:}

	\begin{freeResponse}
	
	\end{freeResponse}
	
\begin{instructorNotes}

\end{instructorNotes}
\end{comment}







\section{Group work:}



%problem 1
\begin{problem}
Evaluate
	\[
	\int_{-\frac{5}{3}}^{-\frac{5}{6}} \frac{\sqrt{36x^2-25}}{x^3} \d x.
	\]
	\begin{freeResponse}
	First notice that
		\begin{align*}
		\sqrt{36x^2-25} &= 5\sqrt{\frac{36x^2}{25} - 1}  \\
		&= 5\sqrt{\left( \frac{6x}{5} \right)^2 - 1}.
		\end{align*}
	So we substitute
		\[
		\frac{6x}{5} = \sec \theta	\qquad	\Longrightarrow	\qquad	x = \frac{5}{6} \sec \theta
		\]
	which gives
		\[
		\d x = \frac{5}{6} \sec \theta \tan \theta \d \theta  .
		\]
	Also, notice that
		\begin{itemize}
		\item when $x = - \frac{5}{3}$:
			\[
			-\frac{5}{3} = \frac{5}{6} \sec \theta \qquad	\Longrightarrow \qquad 	\sec \theta = 2 	\qquad 	\Longrightarrow 	\qquad	\theta = \frac{2\pi}{3}
			\]
			
		\item and when $x=-\frac{5}{6}$:\
			\[
			-\frac{5}{6} = \frac{5}{6} \sec \theta	\qquad	\Longrightarrow	\qquad	\sec \theta = -1	\qquad	\Longrightarrow	\qquad	\theta = \pi.
			\]
		\end{itemize}

	Therefore
		\begin{align*}
		\int_{-\frac{5}{3}}^{-\frac{5}{6}} \frac{\sqrt{36x^2-25}}{x^3} \d x 
		&= 5 \int_{\frac{2\pi}{3}}^{\pi} \frac{\sqrt{\sec^2 \theta - 1}}{\left( \frac{5}{6} \sec \theta \right)^3} \left( \frac{5}{6} \sec \theta \tan \theta \right) \d \theta  \\
		&= 5 \cdot \left(\frac{6}{5} \right)^2 \int_{\frac{2\pi}{3}}^{\pi} \frac{|\tan \theta| \tan \theta}{\sec^2 \theta} \d \theta .
		\end{align*}
	Now, notice that $\tan \theta < 0$ whenever $\frac{2\pi}{3} \leq \theta \leq \pi$.  
	So $|\tan \theta| = - \tan \theta$.
	We continue:
		\begin{align*}
		5 \cdot \left(\frac{6}{5} \right)^2 \int_{\frac{2\pi}{3}}^{\pi} \frac{|\tan \theta| \tan \theta}{\sec^2 \theta} \d \theta
		&= - \frac{36}{5} \int_{\frac{2\pi}{3}}^{\pi} \frac{\tan^2 \theta}{\sec^2 \theta} \d \theta  \\
		&= - \frac{36}{5} \int_{\frac{2\pi}{3}}^{\pi} \frac{\sin^2 \theta}{\cos^2 \theta} \cdot \frac{\cos^2 \theta}{1} \d \theta  \\
		&= - \frac{36}{5} \int_{\frac{2\pi}{3}}^{\pi} \sin^2 \theta \d \theta  \\
		&= - \frac{36}{5} \int_{\frac{2\pi}{3}}^{\pi} \frac{1}{2} \left( 1 - \cos (2\theta) \right) \d \theta  \\
		&= - \frac{18}{5} \eval{\theta - \frac{1}{2} \sin (2\theta)}_{\frac{2\pi}{3}}^{\pi}  \\
		&= - \frac{18}{5} \left[ \left( \pi - 0 \right) - \left( \frac{2\pi}{3} + \frac{\sqrt{3}}{4} \right) \right]  
		\qquad {\color{red} \sin \left( \frac{4\pi}{3} \right) = -\frac{\sqrt{3}}{2} }\\
		&= - \frac{18}{5} \left( \frac{\pi}{3} - \frac{\sqrt{3}}{4} \right).
		\end{align*}
	\end{freeResponse}
	
\end{problem}

\begin{instructorNotes}

\end{instructorNotes}







%problem 2
\begin{problem}
Evaluate
	\[
	\int \frac{\d x}{\left( x^2 - 6x + 11 \right)^2}.
	\]
	\begin{freeResponse}
	We begin by completing the square in the denominator
		\[
		x^2 - 6x - 11 = x^2 - 6x + 9 + 2 = (x-3)^2 + 2.
		\]
	We then have that
		\begin{align*}
		\int \frac{\d x}{\left( x^2 - 6x + 11 \right)^2} 
		&= \int \frac{1}{((x-3)^2 + 2)^2} \d x  \\
		&= \frac{1}{4} \int \frac{1}{\left( \frac{(x-3)^2}{2} + 1 \right)^2} \d x  \\
		&= \frac{1}{4} \int \frac{1}{\left( \left( \frac{x-3}{\sqrt{2}} \right)^2 + 1 \right)^2} \d x.
		\end{align*}
	So we substitute
		\begin{equation}\label{substitution1}
		\frac{x-3}{\sqrt{2}} = \tan \theta	\qquad	\Longrightarrow	\qquad	x = \sqrt{2} \tan \theta + 3
		\end{equation}
	and then
		\[
		\d x = \sqrt{2} \sec^2 \theta \d \theta.
		\]
	Continuing with the integral
		\begin{align*}
		\frac{1}{4} \int \frac{1}{\left( \left( \frac{x-3}{\sqrt{2}} \right)^2 + 1 \right)^2} \d x
		&= \frac{1}{4} \int \frac{1}{\left( \tan^2 \theta + 1 \right)^2} \sqrt{2} \sec^2 \theta \d \theta  \\
		&= \frac{\sqrt{2}}{4} \int \frac{1}{\sec^2 \theta} \d \theta  \\
		&= \frac{\sqrt{2}}{4} \int \cos^2 \theta \d \theta  \\
		&= \frac{\sqrt{2}}{4} \int \frac{1}{2} \left( 1 + \cos(2\theta) \right) \d \theta  \\
		&= \frac{\sqrt{2}}{8} \left( \theta + \frac{1}{2} \sin(2\theta) \right) + C .
		\end{align*}
	Now all that is left to do is to reverse-substitute for $\theta$.  
	First, from equation \eqref{substitution1} we have that
		\[
		\theta = \arctan \left( \frac{x-3}{\sqrt{2}} \right).
		\]
	Now, we again use equation \eqref{substitution1} along with Pythagorean's Theorem to construct the following triangle.
		%\begin{image}
		%\includegraphics[trim= 170 420 250 180]{Figure7-4-1.pdf}
		%\end{image}
	Then we have that
		\[
		\sin(2\theta) = 2 \sin(\theta) \cos(\theta) = 2 \cdot \frac{x-3}{\sqrt{(x-3)^2+2}} \cdot \frac{\sqrt{2}}{\sqrt{(x-3)^2+2}}.
		\]
	Thus
		\[
		\int \frac{\d x}{\left( x^2 - 6x + 11 \right)^2} = \frac{\sqrt{2}}{8} \left( \arctan \left( \frac{x-3}{\sqrt{2}} \right) + \frac{\sqrt{2}(x-3)}{(x-3)^2+2} \right) + C .
		\]
	\end{freeResponse}
		
\end{problem}

\begin{instructorNotes}

\end{instructorNotes}







%problem 3
\begin{problem}
Evaluate
	\[
	\int \frac{x^2}{\sqrt{4x-x^2}} \d x.
	\]
	\begin{freeResponse}
	Again, we begin by completing the square in the denominator, and then factoring
		\begin{align*}
		4x-x^2 &= -(x^2-4x)  \\
		&= -(x^2-4x+4) + 4  \\
		&= -(x-2)^2 + 4  \\
		&= 4 \left( - \frac{(x-2)^2}{4} + 1 \right)  \\
		&= 4 \left( 1 - \left( \frac{x-2}{2} \right)^2 \right).
		\end{align*}
	So
		\begin{align*}
		\int \frac{x^2}{\sqrt{4x-x^2}} \d x &= \int \frac{x^2}{\sqrt{4 \left( 1 - \left( \frac{x-2}{2} \right)^2 \right)}} \d x  \\
		&= \frac{1}{2} \int \frac{x^2}{\sqrt{1 - \left( \frac{x-2}{2} \right)^2}} \d x.
		\end{align*}
	We make the substitution
		\begin{equation}\label{substitution2}
		\frac{x-2}{2} = \sin \theta	\qquad	\Longrightarrow	\qquad	x = 2\sin \theta + 2
		\end{equation}
	which gives
		\[
		\d x = 2 \cos \theta \d \theta.
		\]
	Continuing with the integral, we have that
		\begin{align*}
		\frac{1}{2} \int \frac{x^2}{\sqrt{1 - \left( \frac{x-2}{2} \right)^2}} \d x
		&= \frac{1}{2} \int \frac{(2 \sin \theta + 2)^2}{\sqrt{1 - \sin^2 \theta}} \cdot 2 \cos \theta \d \theta  \\
		&= \int (2\sin \theta + 2)^2 \d \theta  \\
		&= \int ( 4 \sin^2 \theta + 8 \sin \theta + 4) \d \theta  \\
		&= \int (2(1-\cos(2\theta)) + 8\sin \theta + 4) \d \theta  \\
		&= \int (6 + 8\sin \theta - 2\cos(2\theta)) \d \theta  \\
		&= 6 \theta - 8 \cos \theta - \sin(2\theta) + C.
		\end{align*}
	Now all that is left to do is to reverse-substitute for $\theta$.  
	First, from equation \eqref{substitution2} we have that
		\[
		\theta = \arcsin \left( \frac{x-2}{2} \right).
		\]
	Now, we again use equation \eqref{substitution2} along with Pythagorean's Theorem to construct the following triangle.
		%\begin{image}
		%\includegraphics[trim= 170 420 250 180]{Figure7-4-2.pdf}
		%\end{image}
	Then we have that
		\begin{align*}
		&\cos \theta = \frac{\sqrt{4-(x-2)^2}}{2}  \\
		&\sin(2\theta) = 2 \sin \theta \cos \theta = 2 \cdot \frac{x-2}{2} \cdot \frac{\sqrt{4-(x-2)^2}}{2}.
		\end{align*}
	Thus
		\[
		\int \frac{x^2}{\sqrt{4x-x^2}} \d x = 6\arcsin \left( \frac{x-2}{2} \right) - 4\sqrt{4-(x-2)^2} - \frac{(x-2)\sqrt{4-(x-2)^2}}{2}.
		\]
	\end{freeResponse}

\end{problem}

\begin{instructorNotes}

\end{instructorNotes}







%problem 4
\begin{problem}
Evaluate
	\[
	\int \frac{e^x}{\sqrt{e^{2x}+9}} \d x.
	\]
	\begin{freeResponse}
	First, notice that
		\[
		\sqrt{e^{2x}+9} = \sqrt{9\left( \frac{e^{2x}}{9} + 1 \right)} = 3\sqrt{\left( \frac{e^x}{3} \right)^2 + 1}.
		\]
	So
		\[
		\int \frac{e^x}{\sqrt{e^{2x}+9}} \d x = \frac{1}{3} \int \frac{e^x}{\sqrt{\left( \frac{e^x}{3} \right)^2 + 1}} \d x.
		\]
	We make the substitution
		\begin{equation}\label{substitution3}
		\frac{e^x}{3} = \tan \theta 	\qquad	\Longrightarrow	\qquad	3\tan \theta = e^x
		\end{equation}
	which gives
		\[
		e^x \d x = 3 \sec^2 \theta \d \theta.
		\]
	Continuing with the integral, we have that
		\begin{align*}
		\frac{1}{3} \int \frac{e^x}{\sqrt{\left( \frac{e^x}{3} \right)^2 + 1}} \d x
		&= \frac{1}{3} \int \frac{1}{\sqrt{\tan^2 \theta + 1}} \cdot 3 \sec^2 \theta \d \theta  \\
		&= \int \sec \theta \d \theta  \\
		&= \ln | \sec \theta + \tan \theta | + C.
		\end{align*}
	Now all that is left to do is to reverse-substitute for $\theta$. 
	We use equation \eqref{substitution3} along with Pythagorean's Theorem to construct the following triangle.
		%\begin{image}
		%\includegraphics[trim= 170 420 250 180]{Figure7-4-3.pdf}
		%\end{image}
	Then we have that
		\begin{align*}
		&\sec \theta = \frac{\sqrt{e^{2x}+9}}{3} \\
		&\tan \theta = \frac{e^x}{3}.
		\end{align*}
	Thus
		\[
		\int \frac{e^x}{\sqrt{e^{2x}+9}} \d x = \ln \left( \frac{\sqrt{e^{2x}+9} + e^x}{3} \right) + C.
		\]
	\end{freeResponse}

\end{problem}

\begin{instructorNotes}

\end{instructorNotes}







%problem 5
\begin{problem}
Evaluate
	\[
	\int \frac{\d x}{x^{\frac{1}{2}} - 9x^{\frac{3}{2}}}.
	\]
	\begin{freeResponse}
		\begin{align*}
		\int \frac{\d x}{x^{\frac{1}{2}} - 9x^{\frac{3}{2}}}
		&= \int \frac{1}{x^{\frac{1}{2}} \left( 1 - 9x \right)} \d x  \\
		&= \int \frac{1}{x^{\frac{1}{2}} \left( 1 - \left( 3x^{\frac{1}{2}} \right)^2 \right)} \d x  \\
		&= \frac{2}{3} \int \frac{1}{1-u^2} \d x 	\qquad	{\color{red}\text{where }u=3x^{\frac{1}{2}}}  \\
		&= \frac{2}{3} \int \frac{1}{1-\sin^2 \theta} \cos \theta \d \theta	\qquad	{\color{red} \text{where } u=\sin \theta}  \\
		&= \frac{2}{3} \int \frac{\cos \theta}{\cos^2 \theta} \d \theta  \\
		&= \frac{2}{3} \int \sec \theta \d \theta  \\
		&= \frac{2}{3} \ln | \sec \theta + \tan \theta | + C  \\
		&= \frac{2}{3} \ln \left| \frac{1}{\sqrt{1-u^2}} + \frac{u}{\sqrt{1-u^2}} \right| + C  \\
		&= \frac{2}{3} \ln \left( \frac{1 + 3\sqrt{x}}{\sqrt{1-9x}} \right) + C
		\end{align*}
		
		%\begin{image}
		%\includegraphics[trim= 170 420 250 180]{Figure7-4-4.pdf}
		%\end{image}
	\end{freeResponse}

\end{problem}

\begin{instructorNotes}

\end{instructorNotes}
















	
	
	
	
	
	
	
	
	

	










								
				
				
	














\end{document} 


















