\documentclass[]{ximera}
%handout:  for handout version with no solutions or instructor notes
%handout,instructornotes:  for instructor version with just problems and notes, no solutions
%noinstructornotes:  shows only problem and solutions

%% handout
%% space
%% newpage
%% numbers
%% nooutcomes

%I added the commands here so that I would't have to keep looking them up
%\newcommand{\RR}{\mathbb R}
%\renewcommand{\d}{\,d}
%\newcommand{\dd}[2][]{\frac{d #1}{d #2}}
%\renewcommand{\l}{\ell}
%\newcommand{\ddx}{\frac{d}{dx}}
%\everymath{\displaystyle}
%\newcommand{\dfn}{\textbf}
%\newcommand{\eval}[1]{\bigg[ #1 \bigg]}

%\begin{image}
%\includegraphics[trim= 170 420 250 180]{Figure1.pdf}
%\end{image}

%add a ``.'' below when used in a specific directory.
\input{../preamble.tex} %% we can turn off input when making a master document

\title{The ratio, root, and comparison tests}  

\begin{document}
\begin{abstract}		\end{abstract}
\maketitle



%Comparison Test Problem
\section{Warm up:}
For each of the following, answer {\bf True} or {\bf False}, and explain why.
	\begin{enumerate}
	
	\item  If $a_n \geq 0$ and $\sum_{n=0}^\infty a_n$ converges, then $\sum_{n=0}^\infty a_n^2$ converges.
	
	\item  If $a_n, b_n \geq 0$ and both $\sum_{n=0}^\infty a_n$ and $\sum_{n=0}^\infty b_n$ converge, then $\sum_{n=0}^\infty a_n b_n$ converges.
	
	\end{enumerate}
	
	\begin{freeResponse}
		\begin{enumerate}
		
		\item  \dfn{True}
		
		Since $\sum_{n=0}^\infty a_n$ converges, $\lim_{n \to \infty} a_n = 0$.  
		So, in particular, there exists an integer $N$ such that $a_k < 1$ for all $k \geq N$.  
		Then for all $k \geq N$, $a_k^2 < a_k$, and therefore we have that
			\[
			\sum_{n=N}^\infty a_n^2 < \sum_{n=N}^\infty a_n.
			\]
		Thus,  by the Comparison Test, $\sum_{n=0}^\infty a_n^2$ is convergent.
		
		
		
		\item  \dfn{True}
		
		Just as in part (a) there exists an integer $N$ such that $a_k < 1$ for all $k \geq N$.  
		Then
			\[
			\sum_{n=N}^\infty a_n b_n < \sum_{n=N}^\infty b_n
			\]
		and thus, by the Comparison Test, $\sum_{n=0}^\infty a_n b_n$ is convergent.
		
		\end{enumerate}
	\end{freeResponse}
	
\begin{instructorNotes}
Show that these series converge formally using the comparison test.
\end{instructorNotes}







\section{Group work:}



%Root and Ratio Test Problem
%problem 1
\begin{problem}
Determine if the following series converge or diverge.
	\begin{enumerate}
	
	\item  $\sum_{n=1}^\infty \frac{(7n+1)^2 \cdot 2^n}{5^n}$
	
	\item  $\sum_{n=1}^\infty a_n$, where $a_{n+1} = \frac{2n+5}{3n-1}$ and $a_1 = 1$.
	
	\item  $\sum_{n=0}^\infty \frac{n^2 + 2n + 1}{3n^2 +1}$
	
	\item  $\sum_{n=2}^\infty \frac{1}{n(\ln n)^2}$
	
	\end{enumerate}
	
	\begin{freeResponse}
		\begin{enumerate}
	
		\item  \dfn{Ratio Test}
			\begin{align*}
			\lim_{n \to \infty} \frac{a_{n+1}}{a_n} 
			&= \lim_{n \to \infty} \left[ \frac{(7(n+1) + 1)^2 \cdot 2^{n+1}}{5^{n+1}} \cdot \frac{5^n}{(7n+1)^2 \cdot 2^n} \right]  \\
			&= \lim_{n \to \infty} \frac{(7n+8)^2 \cdot 2}{5 \cdot (7n+1)^2}  \\
			&= \frac{49 \cdot 2}{49 \cdot 5} = \frac{2}{5}.
			\end{align*}
		Thus, since $\lim_{n \to \infty} \frac{a_{n+1}}{a_n} < 1$, this series \boxed{converges}.  
		
		
	
		\item  \dfn{Ratio Test}
		
		Even though the terms in this series look a little weird, this is set up perfectly for the Ratio Test:
			\[
			\lim_{n \to \infty} \frac{a_{n+1}}{a_n} = \lim_{n \to \infty} \frac{2n+5}{3n-1} = \frac{2}{3}.
			\]
		Thus, since $\lim_{n \to \infty} \frac{a_{n+1}}{a_n} < 1$, this series \boxed{converges}.  
		
		
	
		\item  \dfn{Divergence Test}
		
		Notice that
			\[
			\lim_{n \to \infty} a_n = \lim_{n \to \infty} \frac{n^2 + 2n + 1}{3n^2 +1} = \frac{1}{3}.
			\]
		Therefore, since $\lim_{n \to \infty} a_n \neq 0$, by the Divergence Test this series \boxed{diverges}.
		
		
	
		\item  \dfn{Integral Test}
		
		First, notice that $f(x) = \frac{1}{x (\ln x)^2}$ is a decreasing and positive function on $[2,\infty)$.
		Then
			\begin{align*}
			\int_2^\infty f(x) \d x 
			&= \int_2^\infty \frac{1}{x (\ln x)^2} \d x  \\
			&= \lim_{b \to \infty} \int_2^b \frac{1}{x (\ln x)^2} \d x  \\
			&= \lim_{b \to \infty} \int_{\ln 2}^{\ln b} u^{-2} \d u  \quad  {\color{red} u = \ln x, \d u = \frac{1}{x} \d x}  \\
			&= \lim_{b \to \infty} \eval{\frac{-1}{u}}_{\ln 2}^{\ln b}  \\
			&= \lim_{b \to \infty} \left( \frac{-1}{\ln b} + \frac{1}{\ln 2} \right)  \\
			&= 0 + \frac{1}{\ln 2} = \frac{1}{\ln 2}.
			\end{align*}
		Therefore, since the above integral converges, the series $\sum_{n=2}^\infty \frac{1}{n(\ln n)^2}$ \boxed{converges} by the Integral Test.
	
		\end{enumerate}
	\end{freeResponse}
	
\end{problem}

\begin{instructorNotes}
Let the students experiment with what tests to use.  
Perhaps give two problems per group.
\end{instructorNotes}







%Root and Ratio Test Problem
%problem 2
\begin{problem}
How many terms are needed to estimate $\sum_{k=1}^\infty \frac{1}{k^2+1}$ to within $10^{-4}$?
What is the estimate for the sum of the series?
	\begin{freeResponse}
	Let $f(x) = \frac{1}{x^2 + 1}$.  
	Note that $f(x)$ is continuous, decreasing, and positive for $x \geq 1$, and $a_k = f(k)$.  
	
	If $S = \sum_{k=1}^\infty \frac{1}{k^2 + 1}$ is the actual value of the sum, and $S_n = \sum_{k=1}^n \frac{1}{k^2 + 1}$ is the $n^{th}$ partisl sum, then the remainder $R_n = S - S_n$ can be bounded above by
		\[
		R_n < \int_n^\infty f(x) \d x.
		\]
	So we need to find $n$ so that
		\begin{equation}\label{inequality}
		\int_n^\infty \frac{1}{x^2 + 1} \d x < 10^{-4}.
		\end{equation}
	We compute
		\begin{align*}
		\int_n^\infty \frac{1}{x^2 + 1} \d x
		&= \lim_{b \to \infty} \int_n^b \frac{1}{x^2+1} \d x  \\
		&= \lim_{b \to \infty} \eval{ \arctan(x) }_n^b  \\
		&= \lim_{b \to \infty} \left( \arctan(b) - \arctan(n) \right)  \\
		&= \frac{\pi}{2} - \arctan(n).
		\end{align*}
	Plugging into equation \eqref{inequality} we see that we want
		\begin{align*}
		&\frac{\pi}{2} - \arctan(n) < 10^{-4}  \\
		\Longrightarrow 	\quad	&\arctan(n) > \frac{\pi}{2} - 10^{-4}  \\
		\Longrightarrow 	\quad	&n > \tan \left( \frac{\pi}{2} - 10^{-4} \right) \approx 9999.999967 \approx 10,000.
		\end{align*}
		
	Therefore, $10,000$ terms will estimate $\sum_{k=1}^\infty \frac{1}{k^2+1}$ to within $10^{-4}$.  
	Using a computer, we can also compute
		\[
		\sum_{k=1}^{10,000} \frac{1}{k^2+1} \approx 1.07657.
		\]
	\end{freeResponse}
		
\end{problem}

\begin{instructorNotes}
This should be a straightforward calculation, but calculators may be needed.
\end{instructorNotes}







%Comparison Test Problem
%problem 3
\begin{problem}
	\begin{enumerate}
	
	\item  Why can we not use the Comparison test with $\sum_{k=1}^\infty \frac{1}{k^2}$ to show that $\sum_{k=1}^\infty \frac{1}{k^2 - 5}$ converges?
	
	\item  Adjust $\sum_{k=1}^\infty \frac{1}{k^2}$ to show that $\sum_{k=1}^\infty \frac{1}{k^2 - 5}$ converges via the Comparison Test.
	
	\item  Give a convergent series we can use in the Limit Comparison Test to show that $\sum_{k=1}^\infty \frac{1}{k^2 - 5}$ converges.  
	
	\end{enumerate}
	
	\begin{freeResponse}
		\begin{enumerate}
		
		\item  We cannot use the Comparison Test here because 
			\[
			\frac{1}{k^2} < \frac{1}{k^2 - 5}
			\]
		for all $k \geq 1$.  So we would just be showing the the series in question is greater than a series which converges, which does not give us any information.
		
		
		
		\item  Notice that
			\[
			\frac{2}{k^2} > \frac{1}{k^2 - 5}
			\]
		for all $k \geq 4$.  
		Since $\sum_{k=1}^\infty \frac{1}{k^2}$ converges, $\sum_{k=1}^\infty \frac{2}{k^2} = 2 \sum_{k=1}^\infty \frac{1}{k^2}$.  
		Thus, $\sum_{k=1}^\infty \frac{2}{k^2}$ converges.
		
		Therefore, the Comparison Test with $\sum_{k=1}^\infty \frac{2}{k^2}$ shows that $\sum_{k=0}^\infty \frac{1}{k^2-5}$ converges.
		
		
		
		\item  For the Limit Comparison Test, we \dfn{can} use $\sum_{k=1}^\infty \frac{1}{k^2}$.  
			\begin{align*}
			\lim_{k \to \infty} \frac{\frac{1}{k^2-5}}{\frac{1}{k^2}}
			&= \lim_{k \to \infty} \frac{k^2}{k^2-5}  \\
			&= 1.
			\end{align*}
			
		Thus, since $\sum_{k=1}^\infty \frac{1}{k^2}$ converges, by the Limit Comparison Test we know that $\sum_{k=0}^\infty \frac{1}{k^2-5}$ converges.
		
		\end{enumerate}
	\end{freeResponse}

\end{problem}

\begin{instructorNotes}
This problem may be done as a quick whole class discussion.  
For (b), use something like $\frac{2}{k^2}$.  
Be sure to determine for which $k$ the inequality will hold.
\end{instructorNotes}







%Comparison Test Problem
%problem 4
\begin{problem}
Determine if the following series converge or diverge.
	\begin{multicols}{2}
	\begin{enumerate}
	
	\item  $\sum_{n=0}^\infty \frac{n^2 + 2n + 1}{3n^3+1}$
	
	\item  $\sum_{n=0}^\infty \frac{n^2+2n+1}{3n^4+1}$
	
	\item  $\sum_{n=0}^\infty \frac{\cos^2 n}{n^3+1}$
	
	\item  $\sum_{n=1}^\infty \left[ \left( 1+\frac{1}{n} \right)^2 e^{-n} \right]$
	
	\end{enumerate}
	\end{multicols}
	
	\begin{freeResponse}
		\begin{enumerate}
		
		\item  Use the \dfn{Limit Comparison Test} with $\sum_{n=1}^\infty \frac{1}{n}$.
			\begin{align*}
			\lim_{n \to \infty} \frac{a_n}{b_n}
			&= \lim_{n \to \infty} \frac{n^2+2n+1}{3n^3+1} \cdot \frac{n}{1}  \\
			&= \frac{1}{3}.
			\end{align*}
			
		Therefore, since $\sum_{n=1}^\infty \frac{1}{n}$ diverges, by the Limit Comparison Test we know that $\sum_{n=0}^\infty \frac{n^2+2n+1}{3n^3+1}$ diverges.
		
		
		
		\item  Use the \dfn{Limit Comparison Test} with $\sum_{n=1}^\infty \frac{1}{n^2}$.
			\begin{align*}
			\lim_{n \to \infty} \frac{a_n}{b_n}
			&= \lim_{n \to \infty} \frac{n^2+2n+1}{3n^4+1} \cdot \frac{n^2}{1}  \\
			&= \frac{1}{3}.
			\end{align*}
			
		Therefore, since $\sum_{n=1}^\infty \frac{1}{n^2}$ converges, by the Limit Comparison Test we know that $\sum_{n=0}^\infty \frac{n^2+2n+1}{3n^4+1}$ converges.
		
		
		
		\item  Use the \dfn{Comparison Test} with $\sum_{n=1}^\infty \frac{1}{n^3}$.  
		
		Since we always have that $0 < \cos^2(n) < 1$, we know that
			\[
			\frac{\cos^2(n)}{n^3 + 1} \leq \frac{1}{n^3+1} < \frac{1}{n^3}.
			\]
		Therefore, by the Comparison Test, we have that $\sum_{n=0}^\infty \frac{\cos^2 n}{n^3+1}$ converges.
		
		
		
		\item  Use the \dfn{Comparison Test} with $\sum_{n=1}^\infty 2 \cdot e^{-n}$.
		
		First, notice that for all $n \geq 3$,
			\[
			\left( 1 + \frac{1}{n} \right)^2 < 2.
			\]		
		Also, notice that $\sum_{n=1}^\infty e^{-n}$ is a convergent geometric series.  
		Therefore $\sum_{n=1}^\infty 2 \cdot e^{-n}$ converges, and so $\sum_{n=1}^\infty \left[ \left( 1+\frac{1}{n} \right)^2 e^{-n} \right]$ converges by the Comparison Test.
		
		\end{enumerate}
	\end{freeResponse}

\end{problem}

\begin{instructorNotes}
These are all done using either the Comparison Test or the Limit Comparison Test.  
Parts (a) and (b) should be done with the limit comparison test (explain why).  
It is important to compare and contrast these two problems.  
Parts (c) and (d) should be done with the Comparison Test (again, explain why).  
The $e^{-n}$ should be treated as a geometric series.
\end{instructorNotes}
















	
	
	
	
	
	
	
	
	

	










								
				
				
	














\end{document} 


















