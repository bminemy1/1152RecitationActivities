\documentclass[]{ximera}
%handout:  for handout version with no solutions or instructor notes
%handout,instructornotes:  for instructor version with just problems and notes, no solutions
%noinstructornotes:  shows only problem and solutions

%% handout
%% space
%% newpage
%% numbers
%% nooutcomes

%I added the commands here so that I would't have to keep looking them up
%\newcommand{\RR}{\mathbb R}
%\renewcommand{\d}{\,d}
%\newcommand{\dd}[2][]{\frac{d #1}{d #2}}
%\renewcommand{\l}{\ell}
%\newcommand{\ddx}{\frac{d}{dx}}
%\everymath{\displaystyle}
%\newcommand{\dfn}{\textbf}
%\newcommand{\eval}[1]{\bigg[ #1 \bigg]}

%\begin{image}
%\includegraphics[trim= 170 420 250 180]{Figure1.pdf}
%\end{image}

%add a ``.'' below when used in a specific directory.
\input{../preamble.tex} %% we can turn off input when making a master document

\title{Working with Taylor series}  

\begin{document}
\begin{abstract}		\end{abstract}
\maketitle



\section{Warm up:}
True or False:  To approximate $\frac{\pi}{3}$, one could substitute $x = \sqrt{3}$ into the Maclaurin series for $\tan^{-1}x$?
	\begin{freeResponse}
	
	\end{freeResponse}
	
\begin{instructorNotes}
%There were no instructor notes for this handout.
\end{instructorNotes}







\section{Group work:}



%problem 1
\begin{problem}
Use power series to evaluate the limit
	\[
	\lim_{x \to 0} \frac{\ln (1+x^2)}{1 - \cos x}
	\]
	
	\begin{freeResponse}
	
	\end{freeResponse}
	
\end{problem}

\begin{instructorNotes}

\end{instructorNotes}







%problem 2
\begin{problem}
Given that
	\[
	f(t) = \int_0^t x^2 \tan^{-1}(x^4) \d x
	\]
approximate $f \left( \frac{1}{3} \right)$ with the first four non-zero terms of a power series.  
Estimate how close this approximation is.
	
	\begin{freeResponse}
	
	\end{freeResponse}
		
\end{problem}

\begin{instructorNotes}

\end{instructorNotes}







%problem 3
\begin{problem}
Identify the function represented by the power series
	\[
	\sum_{k=0}^\infty \frac{k(k-1)x^4}{7^k}
	\]
	
	\begin{freeResponse}
	
	\end{freeResponse}

\end{problem}

\begin{instructorNotes}

\end{instructorNotes}







%problem 4
\begin{problem}
Use power series to determine a (series) solution to the initial value problem
	\[
	y'' - xy' + y = 0 	\qquad	y(0) = 1 	\qquad	y'(0) = 0
	\]
	
	\begin{freeResponse}
	
	\end{freeResponse}

\end{problem}

\begin{instructorNotes}

\end{instructorNotes}
















	
	
	
	
	
	
	
	
	

	










								
				
				
	














\end{document} 


















