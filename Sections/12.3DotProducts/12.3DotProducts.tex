\documentclass[]{ximera}
%handout:  for handout version with no solutions or instructor notes
%handout,instructornotes:  for instructor version with just problems and notes, no solutions
%noinstructornotes:  shows only problem and solutions

%% handout
%% space
%% newpage
%% numbers
%% nooutcomes

%I added the commands here so that I would't have to keep looking them up
%\newcommand{\RR}{\mathbb R}
%\renewcommand{\d}{\,d}
%\newcommand{\dd}[2][]{\frac{d #1}{d #2}}
%\renewcommand{\l}{\ell}
%\newcommand{\ddx}{\frac{d}{dx}}
%\everymath{\displaystyle}
%\newcommand{\dfn}{\textbf}
%\newcommand{\eval}[1]{\bigg[ #1 \bigg]}

%\begin{image}
%\includegraphics[trim= 170 420 250 180]{Figure1.pdf}
%\end{image}

%add a ``.'' below when used in a specific directory.
\input{../preamble.tex} %% we can turn off input when making a master document

\title{Dot products}  

\begin{document}
\begin{abstract}		\end{abstract}
\maketitle



\section{Warm up:}
If $\vec{u} = \hat{\imath} - 2 \hat{\jmath}$ and $\vec{v} = 3 \hat{\imath} + 4 \hat{k}$, find $\vec{u} \cdot \vec{v}$.
	\begin{freeResponse}
	Note that these vectors are in $\mathbb{R}^3$ and not $\mathbb{R}^2$.
	\[
	\vec{u} \cdot \vec{v} = (1 \cdot 3) + (-2 \cdot 0) + (0 \cdot 4) = \boxed{3}.
	\]
	\end{freeResponse}
	
\begin{instructorNotes}
Make sure that students realize that $\vec{u} = \langle 3,4,0 \rangle$ and not $\langle 3,4 \rangle$.
\end{instructorNotes}







\section{Group work:}



%problem 1
\begin{problem}
Suppose that the deli at the Tiny Sparrow grocery store sells roast beef for $ \$ 9$ per pound, turkey for $ \$ 4$ per pound, salami for $\$ 5$ per pound, and ham for $\$ 7$ per pound.  
For lunches this week, Sam the sandwhich maker buys $1.5$ pounds of roast beef, $2$ pounds of turkey, no salami, and half a pound of ham.  
How can you use a dot product to compute Sam's total bill from the deli?
	\begin{freeResponse}
	The cost vector is
		\[
		\vec{c} = \langle 9,4,5,7 \rangle.
		\]
	The vector for Sam's order is
		\[
		\vec{o} = \left\langle \frac{3}{2}, 2, 0, \frac{1}{2} \right\rangle.
		\]
	Then Sam's bill is
		\[
		\vec{c} \cdot \vec{o} = 9(1.5) + 4(2) + 5(0) + 7(0.5) = 13.5 + 8 + 0 + 3.5 = \boxed{25}.
		\]
	\end{freeResponse}
	
\end{problem}

\begin{instructorNotes}

\end{instructorNotes}







%problem 2
\begin{problem}
Find the work done by a constant force of $10 \hat{\imath} + 18 \hat{\jmath} - 6\hat{k}$ that moves an object up a ramp from $(2,3,7)$ to $(4,9,15)$.  
Assume that distance is in feet and force in pounds.  
Also, find the angle between the force and the ramp.
	\begin{freeResponse}
	First, let $\vec{F} = \langle 10,18,-7 \rangle$.  
	Also, the vector from $(2,3,7)$ to $(4,9,15)$ is $\langle 2,6,8 \rangle$.  
	Let $\vec{d} = \langle 2,6,8 \rangle$.  
	Then the work done by the force is
		\[
		\vec{F} \cdot \vec{d} = 10 \cdot 2 + 18 \cdot 6 - 6 \cdot 8 = \boxed{80 \, ft \cdot lb}.
		\]
		
	To calculate the angle, we compute
		\[
		\cos \theta = \frac{\vec{F} \cdot \vec{d}}{\| \vec{F} \| \| \vec{d} \|} = \frac{80}{\sqrt{460}\sqrt{104}}.
		\]
	and so
		\[
		\theta = \boxed{\cos^{-1} \left(  \frac{80}{\sqrt{460}\sqrt{104}} \right) \approx 1.196 \text{ radians}}
		\]
	\end{freeResponse}
		
\end{problem}

\begin{instructorNotes}

\end{instructorNotes}







%problem 3
\begin{problem}
Find a vector (in the $xy$-plane) with length $4$ that makes a $\frac{\pi}{3}$ radian angle with the vector $\langle 3,4 \rangle$.
	\begin{freeResponse}
	Let $\vec{v} = \langle a,b \rangle$ denote a vector that we are looking for, and let $\vec{u} = \langle 3,4 \rangle$.  
	First note that
		\[
		\| \vec{u} \| = \sqrt{9 + 16} = 5.
		\]
	So
		\[
		\vec{u} \cdot \vec{v} = \| \vec{u} \| \| \vec{v} \| \cos \left( \frac{\pi}{3} \right) = 5 \cdot 4 \cdot \frac{1}{2} = 10.
		\]
	Then we have the following two equations:
		\begin{align}
		&10 = \vec{u} \cdot \vec{v} = 3a + 4b  \label{eqn1}  \\
		&16 = \| \vec{v} \|^2 = a^2 + b^2  \label{eqn2}  .
		\end{align}
	Solving equation \eqref{eqn1} for $a$ gives us
		\[
		a = \frac{10 - 4b}{3}.
		\]
	Plugging this into equation \eqref{eqn2} yields
		\begin{align*}
		\left( \frac{10-4b}{3} \right)^2 + b^2 &= 16  \\
		(10-4b)^2 + 9b^2 &= 144  \\
		16b^2 -80b + 100 + 9b^2 &= 144  \\
		25b^2 - 80b - 44 &= 0
		\end{align*}
	Using the quadratic formula gives
		\begin{align*}
		b &= \frac{80 \pm \sqrt{(-80)^2 - 4(25)(-44)}}{2(25)}  \\
		&= \frac{80 \pm \sqrt{10800}}{50}  \\
		&= \frac{80 \pm 60\sqrt{3}}{50}  \\
		&= \frac{8 \pm 6\sqrt{3}}{5}.
		\end{align*}
		
	We can choose either value for $b$.  
	Choosing $b = \frac{8 + 6 \sqrt{3}}{5}$ gives a value of $a = \frac{10 - 4 \left( \frac{8+6\sqrt{3}}{5} \right)}{3}$.
	Thus,
		\[
		\vec{v} = \boxed{ \left\langle \frac{10 - 4 \left( \frac{8+6\sqrt{3}}{5} \right)}{3} ,  \frac{8 + 6 \sqrt{3}}{5} \right\rangle }
		\]
	\end{freeResponse}

\end{problem}

\begin{instructorNotes}
The students need to assimilate a lot of information in this problem.  
They need to ``name'' the unknown vector (say $\langle a, b \rangle$).  
Then, they need to realize that both $\langle 3,4 \rangle \cdot \langle 3,4 \rangle = 3a + 4b$ and that $\bigr| \langle 3,4 \rangle \bigr| \cdot \bigr| \langle a,b \rangle \bigr| \cos \left( \frac{\pi}{3} \right) = 5 \cdot 4 \cdot \frac{1}{2}$, giving $3a + 4b = 10$.  
Lastly, they also need to realize that $a^2 + b^2 = 16$.  
A picture illustrating that there could be two such vectors would be helpful.
\end{instructorNotes}








%problem 4
\begin{problem}
Answer the following questions about $\text{proj}_v u$.
	\begin{enumerate}
	\item  Is $\text{proj}_v u$ a vector of the form $c \vec{v}$ or $c \vec{u}$ (where $c$ is a real number)?  
	ie, is $\text{proj}_v u$ parallel to $\vec{u}$ or $\vec{v}$?  
	
	\item  If $\vec{u} = 5 \hat{\imath} + 6 \hat{\jmath} - 3 \hat{k}$ and $\vec{v} = 2 \hat{\imath} - 4 \hat{\jmath} + 4 \hat{k}$, find $\text{proj}_v u$.
	
	\item  For $\vec{u}$ and $\vec{v}$ from part (b), write $\vec{u}$ as the sum of two perpendicular vectors, one of which is parallel to $\vec{v}$.  
	\end{enumerate}
	
	\begin{freeResponse}
	\begin{enumerate}
	\item  $\boxed{c \vec{v}}$
	
	\item  
		\begin{align*}
		\text{proj}_v u 
		&= \frac{\vec{u} \cdot \vec{v}}{\vec{v} \cdot \vec{v}} \vec{v}  \\
		&= \frac{10-24-12}{4+16+16} \langle 2,-4,4 \rangle  \\
		&= \boxed{- \frac{13}{18} \langle 2,-4,4 \rangle  }
		\end{align*}
	
	
	
	\item  
		%\begin{image}
		%\includegraphics[trim= 170 420 250 180]{Figure12-3-1.pdf}
		%\end{image}
		
	The vector which is parallel to $\vec{v}$ is 
		\[
		\text{proj}_v u = \boxed{\left\langle - \frac{13}{9}, \frac{26}{9}, - \frac{26}{9} \right\rangle}
		\]
	The vector which is orthogonal to $\vec{v}$ is
		\begin{align*}
		\vec{u} - \text{proj}_v u
		&= \left\langle 5,6,-3 \right\rangle - \left\langle - \frac{13}{9}, \frac{26}{9}, - \frac{26}{9} \right\rangle  \\
		&= \boxed{\left\langle \frac{58}{9}, \frac{28}{9}, - \frac{1}{9} \right\rangle}
		\end{align*}
	And, clearly, $\text{proj}_v u + (\vec{u} - \text{proj}_v u) = \vec{u}$.  
	
	\end{enumerate}
	
	\end{freeResponse}

\end{problem}

\begin{instructorNotes}

\end{instructorNotes}
















	
	
	
	
	
	
	
	
	

	










								
				
				
	














\end{document} 


















