\documentclass[]{ximera}
%handout:  for handout version with no solutions or instructor notes
%handout,instructornotes:  for instructor version with just problems and notes, no solutions
%noinstructornotes:  shows only problem and solutions

%% handout
%% space
%% newpage
%% numbers
%% nooutcomes

%I added the commands here so that I would't have to keep looking them up
%\newcommand{\RR}{\mathbb R}
%\renewcommand{\d}{\,d}
%\newcommand{\dd}[2][]{\frac{d #1}{d #2}}
%\renewcommand{\l}{\ell}
%\newcommand{\ddx}{\frac{d}{dx}}
%\everymath{\displaystyle}
%\newcommand{\dfn}{\textbf}
%\newcommand{\eval}[1]{\bigg[ #1 \bigg]}

%\begin{image}
%\includegraphics[trim= 170 420 250 180]{Figure1.pdf}
%\end{image}

%add a ``.'' below when used in a specific directory.
\input{../preamble.tex} %% we can turn off input when making a master document

\title{Trigonometric integrals}  

\begin{document}
\begin{abstract}		\end{abstract}
\maketitle



\begin{comment}
\section{Warm up:}

	\begin{freeResponse}
	
	\end{freeResponse}
	
\begin{instructorNotes}

\end{instructorNotes}
\end{comment}







\section{Group work:}



%problem 1
\begin{problem}
Evaluate the following integrals
	\begin{enumerate}
	
	\item  $\int \tan^{23} x \sec^6 x \d x$
	\begin{freeResponse}
	
	\end{freeResponse}
	
	
	
	\item  $\int \tan^2 x \sec x \d x$ \qquad {\color{red} Hint:  $\int \sec x \d x = \ln | \sec x \tan x| + C$}
	\begin{freeResponse}
	
	\end{freeResponse}
	
	
	
	\item  $\int \tan^2 x \sin x \d x$
	\begin{freeResponse}
	
	\end{freeResponse}
	
	\end{enumerate}
	
\end{problem}

\begin{instructorNotes}
All three parts involve standard strategies learned in the online lessons.  
For each, split the problems between the groups.  
Then discuss each problem as a class, getting input from the group(s) that worked on that problem.
\end{instructorNotes}







%problem 2
\begin{problem}
Evaluate
	\[
	\int_{- \pi}^0 \sqrt{1 - \cos^2 x} \d x.
	\]
	\begin{freeResponse}
	
	\end{freeResponse}
		
\end{problem}

\begin{instructorNotes}
You may want to do this problem as a whole class - perhaps play-acting by claiming that it is equal to $\int_{-\pi}^0 \sin x \d x$ rather than $\int_{-\pi}^0 \left| \sin x \right|  \d x$
\end{instructorNotes}







\begin{comment}
%problem 3
\begin{problem}

	\begin{freeResponse}
	
	\end{freeResponse}

\end{problem}

\begin{instructorNotes}

\end{instructorNotes}
\end{comment}
















	
	
	
	
	
	
	
	
	

	










								
				
				
	














\end{document} 


















