\documentclass[]{ximera}
%handout:  for handout version with no solutions or instructor notes
%handout,instructornotes:  for instructor version with just problems and notes, no solutions
%noinstructornotes:  shows only problem and solutions

%% handout
%% space
%% newpage
%% numbers
%% nooutcomes

%I added the commands here so that I would't have to keep looking them up
%\newcommand{\RR}{\mathbb R}
%\renewcommand{\d}{\,d}
%\newcommand{\dd}[2][]{\frac{d #1}{d #2}}
%\renewcommand{\l}{\ell}
%\newcommand{\ddx}{\frac{d}{dx}}
%\everymath{\displaystyle}
%\newcommand{\dfn}{\textbf}
%\newcommand{\eval}[1]{\bigg[ #1 \bigg]}

%\begin{image}
%\includegraphics[trim= 170 420 250 180]{Figure1.pdf}
%\end{image}

%add a ``.'' below when used in a specific directory.
\input{../preamble.tex} %% we can turn off input when making a master document

\title{Taylor series}  

\begin{document}
\begin{abstract}		\end{abstract}
\maketitle



\section{Warm up:}
Find the Taylor series for:  
	\begin{enumerate}
	\item  $27x^2 - 3x + 17$ about $a=1$.  
	\item  $\sin(2x)$ about $a = \frac{\pi}{8}$.  
	\end{enumerate}
	
	\begin{freeResponse}
	\begin{enumerate}
	\item  
	Let $f(x) = 27x^2-3x+17$.  Then
		\begin{align*}
		&f(1) = 27 - 3 + 17 = 41  \\
		&f'(x) = 54x - 3 	\qquad \Longrightarrow 	\qquad	f'(1) = 54-3=51  \\
		&f''(x) = 54 		\qquad	\Longrightarrow	\qquad f''(1) = 54  \\
		&f^{(3)}(x) = 0 	\qquad	\Longrightarrow	\qquad f^{(3)}(1) = 0  \\
		&\qquad \vdots 	\qquad	\qquad	\qquad \qquad \qquad	\vdots  \\
		&f^{(n)}(x) = 0	\qquad	\Longrightarrow	\qquad	f^{(n)}(1) = 0.
		\end{align*}
	So
		\[
		f(x) = \sum_{k=0}^\infty \frac{f^{(k)}(a)}{k!}(x-a)^k = \boxed{41 + 51(x-1) + \frac{54}{2!}(x-1)^2}
		\]
	Lastly, note that if you multiply this out then you will get back the original polynomial.
	
	
	
	\item Let $f(x) = \sin(2x)$.  Then
		\begin{align*}
		&f \left( \frac{\pi}{8} \right) = \sin \left( \frac{2\pi}{8} \right) = \frac{\sqrt{2}}{2}  \\
		&f'(x) = 2 \cos(2x) 	\qquad \Longrightarrow 	\qquad	f'\left( \frac{\pi}{8} \right) = 2 \cdot \frac{\sqrt{2}}{2} = \sqrt{2}  \\
		&f''(x) = -4 \sin(2x) 		\qquad	\Longrightarrow	\qquad f''\left( \frac{\pi}{8} \right) = -4 \cdot \frac{\sqrt{2}}{2} = -2 \sqrt{2}  \\
		&f^{(3)}(x) = -8 \cos(2x)	\qquad	\Longrightarrow	\qquad f^{(3)}\left( \frac{\pi}{8} \right) = -8 \cdot \frac{\sqrt{2}}{2} = -4 \sqrt{2}  \\
		&f^{(4)}(x) = 16 \cos(2x) \qquad 	\Longrightarrow	\qquad f^{(4)}\left( \frac{\pi}{8} \right) =16 \cdot \frac{\sqrt{2}}{2}  = 8 \sqrt{2}.
		\end{align*}
	Continuing this pattern, we see that
		\[
		f^{(k)} \left( \frac{\pi}{8} \right) = (-1)^{\left\lceil \frac{k}{2} \right\rceil } 2^{k-1} \sqrt{2}
		\]
	where $\left\lceil \frac{k}{2} \right\rceil$ denotes the smallest integer greater than $\frac{k}{2}$.  
	So, for example, $\left\lceil \frac{1}{2} \right\rceil = 1$, $\left\lceil \frac{2}{2} \right\rceil = 1$, $\left\lceil \frac{3}{2} \right\rceil = 2$, and so on.  
	
	So from here we have that the Taylor series for $f(x)$ is
		\[
		\boxed{\sum_{k=0}^\infty \frac{(-1)^{\left\lceil \frac{k}{2} \right\rceil } 2^{k-1} \sqrt{2}}{k!} \left( x - \frac{\pi}{8} \right)^k}
		\]
	
	\end{enumerate}
	\end{freeResponse}
	
\begin{instructorNotes}
Here, they need to compute the Taylor series by computing derivatives and recognizing patterns.  

Part (a) is an opportunity to show the students that a polynomial is already a Maclaurin series.  
Use derivatives to figure out the Taylor series about $a=1$, and then show them that the answer simplifies back to the original problem.  

Part (b) is a continuation from the previous recitation where they already found the approximating polynomial.  
The students may have a problem finding the pattern in the derivatives.
\end{instructorNotes}







\section{Group work:}



%problem 1
\begin{problem}
Find a power series (and interval of convergence) for each of the following functions
	\begin{multicols}{2}
	\begin{enumerate}
	\item  $f(x) = x^3 \sin(x^5)$
	\item  $f(x) = \frac{1}{(1+x)^4}$
	\item  $f(x) = \frac{1}{(3-5x^2)^4}$
	\item  $f(x) = \sin^{-1}(x^5)$
	\end{enumerate}
	\end{multicols}
	
	\begin{freeResponse}
	\begin{enumerate}
	\item
	
	\item
	
	\item
	
	\item
	
	\end{enumerate}
	\end{freeResponse}
	
\end{problem}

\begin{instructorNotes}
Students should use the known Maclaurin series in various ways.  
You might want to give the hint in part (d) that $\ddx \sin^{-1}(x) = \frac{1}{\sqrt{1-x^2}}$.  
\end{instructorNotes}







%problem 2
\begin{problem}
Find a function (closed expression) for the following series and the interval on which the function and the series are equal.
	\[
	x + x^4 + \frac{1}{2} x^7 + \frac{1}{6} x^{10} + \frac{1}{24} x^{13} + \hdots
	\]
	\begin{freeResponse}
	
	\end{freeResponse}
		
\end{problem}

\begin{instructorNotes}
The students need to rewrite $f(x)$ in summation notation (factoring out an $x$) and seeing the series for $xe^{x^3}$.  
\end{instructorNotes}







%problem 3
\begin{problem}
Compute the sum of the following series ({\it Hint:  You should use Taylor series.})
	\begin{enumerate}
	\item  $1 - \ln 2 + \frac{(\ln 2)^2}{2!} - \frac{(\ln 2)^3}{3!} + \hdots$
	\item  $3 + \frac{9}{2!} + \frac{27}{3!} + \frac{81}{4!} + \hdots$
	\end{enumerate}
	
	\begin{freeResponse}
	\begin{enumerate}
	\item  
	\item  
	\end{enumerate}
	\end{freeResponse}

\end{problem}

\begin{instructorNotes}
The goal here is for students to realize that Taylor series gives them a tool for finding the exact sum of a series.
\end{instructorNotes}
















	
	
	
	
	
	
	
	
	

	










								
				
				
	














\end{document} 


















