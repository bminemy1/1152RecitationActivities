\documentclass[]{ximera}
%handout:  for handout version with no solutions or instructor notes
%handout,instructornotes:  for instructor version with just problems and notes, no solutions
%noinstructornotes:  shows only problem and solutions

%% handout
%% space
%% newpage
%% numbers
%% nooutcomes

%I added the commands here so that I would't have to keep looking them up
%\newcommand{\RR}{\mathbb R}
%\renewcommand{\d}{\,d}
%\newcommand{\dd}[2][]{\frac{d #1}{d #2}}
%\renewcommand{\l}{\ell}
%\newcommand{\ddx}{\frac{d}{dx}}
%\everymath{\displaystyle}
%\newcommand{\dfn}{\textbf}
%\newcommand{\eval}[1]{\bigg[ #1 \bigg]}

%\begin{image}
%\includegraphics[trim= 170 420 250 180]{Figure1.pdf}
%\end{image}

%add a ``.'' below when used in a specific directory.
\input{../preamble.tex} %% we can turn off input when making a master document

\title{Direction fields and Euler's method}  

\begin{document}
\begin{abstract}		\end{abstract}
\maketitle



\begin{comment}
\section{Warm up:}

	\begin{freeResponse}
	
	\end{freeResponse}
	
\begin{instructorNotes}

\end{instructorNotes}
\end{comment}







\section{Group work:}



%problem 1
\begin{problem}
	
	\begin{enumerate}
	\item  The following is a direction field for the differential equation $\dd[y]{x} = y^2 - x^2$.
	
		%\begin{image}
		%\includegraphics[trim= 170 420 250 180]{Figure8-2-1.pdf}
		%\end{image}
		
		Sketch the solution such that $y\left( \frac{1}{2} \right) = 1$.
	\begin{freeResponse}
	
	\end{freeResponse}
	
	
	
	\item  Use Euler's Method to give a numerical estimate to the solution of the differential equation $y' = y^2 - t^2$ at $y(2)$ that goes through the point $\left( \frac{1}{2}, 1 \right)$.  
	Use $\Delta t = 0.5$.
	\begin{freeResponse}
	
	\end{freeResponse}
	\end{enumerate}
	
\end{problem}

\begin{instructorNotes}
The major point here is that (a) and (b) are the same problem, presented with two different representations.
\end{instructorNotes}







%problem 2
\begin{problem}
Describe why the following direction field could be the direction field for the differential equation 
	\[
	\dd[y]{t} y \cos(t)
	\]
but \dfn{not} for 
	\[
	\dd[y]{t} = y \sin(t) 	\qquad	\text{or}	\qquad	\dd[y]{t} = t \cos(y).
	\]
	
	%\begin{image}
	%\includegraphics[trim= 170 420 250 180]{Figure8-2-2.pdf}
	%\end{image}
	\begin{freeResponse}
	
	\end{freeResponse}
		
\end{problem}

\begin{instructorNotes}
Students should examine when $t$ varies across quadrants, combined with the sign of $y$ at points $(t,y)$ (with $y'=t \cos y$, $y$ is going through ``quadrants'').  Checking where $y'=0$ can reveal why the other two differential equations are not satisfied.  
This could all just be a whole class discussion with the instructor bringing up strategies.
\end{instructorNotes}







%problem 3
\begin{problem}
Match each of the following differential equations with a corresponding direction field:
	\begin{multicols}{2}
	\begin{enumerate}
	\item[i.]	$y' = \frac{t}{2+y}$
	\item[ii.]	$y' = \cos(t+y)$
	\item[iii.]	$y' = 1+y^2$
	\item[iv.]	$y' = ty$
	\end{enumerate}
	\end{multicols}
	
	%\begin{image}
	%\includegraphics[trim= 170 420 250 180]{Figure8-2-3.pdf}
	%\end{image}
	
	\begin{freeResponse}
	
	\end{freeResponse}

\end{problem}

\begin{instructorNotes}
Several strategies exist.  
Make sure to ask what special quality direction fields of autonomous differential equations have.  
Depending on time, this also could all be done as a whole class.
\end{instructorNotes}
















	
	
	
	
	
	
	
	
	

	










								
				
				
	














\end{document} 


















