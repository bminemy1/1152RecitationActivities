\documentclass[]{ximera}
%handout:  for handout version with no solutions or instructor notes
%handout,instructornotes:  for instructor version with just problems and notes, no solutions
%noinstructornotes:  shows only problem and solutions

%% handout
%% space
%% newpage
%% numbers
%% nooutcomes

%I added the commands here so that I would't have to keep looking them up
%\newcommand{\RR}{\mathbb R}
%\renewcommand{\d}{\,d}
%\newcommand{\dd}[2][]{\frac{d #1}{d #2}}
%\renewcommand{\l}{\ell}
%\newcommand{\ddx}{\frac{d}{dx}}
%\everymath{\displaystyle}
%\newcommand{\dfn}{\textbf}
%\newcommand{\eval}[1]{\bigg[ #1 \bigg]}

%\begin{image}
%\includegraphics[trim= 170 420 250 180]{Figure1.pdf}
%\end{image}

%add a ``.'' below when used in a specific directory.
\input{../preamble.tex} %% we can turn off input when making a master document

\title{Infinite series}  

\begin{document}
\begin{abstract}		\end{abstract}
\maketitle



\begin{comment}
\section{Warm up:}

	\begin{freeResponse}
	
	\end{freeResponse}
	
\begin{instructorNotes}

\end{instructorNotes}
\end{comment}







\section{Group work:}



%problem 1
\begin{problem}
Determine if the following series converge or diverge.  If they converge, find the sum.
	\begin{enumerate}
	
	\item  $e + 1 + e^{-1} + e^{-2} + e^{-3} + \hdots$
	\begin{freeResponse}
		\begin{align*}
		e + 1 + e^{-1} + e^{-2} + e^{-3} + \hdots
		&= e + \sum_{k = 0}^\infty e^{-k}  \\
		&= e + \sum_{k=0}^\infty \left( e^{-1} \right)^k 	\quad	{\color{red}\text{geometric series, }r = e^{-1} < 1}  \\
		&= e + \frac{1}{1-e^{-1}}.
		\end{align*}
	Therefore, this series converges to $\left( e + \frac{1}{1-e^{-1}} \right)$.  
	\end{freeResponse}
	
	
	
	\item  $\sum_{k=0}^{99} 2^k + \sum_{k=100}^\infty \frac{1}{2^k}$
	\begin{freeResponse}
	Let us analyze the two different summands in this problem:
		\begin{enumerate}
		\item[(i)]  $\sum_{k=0}^{99} 2^k$
		
		This is a finite sum from a geometric sequence, and so its sum is 
			\[
			\frac{a(1-r^n)}{1-r}.
			\]
		Thus,
	  		\[
	  		\sum_{k=0}^{99} 2^k = \frac{1(1-2^{100})}{1-2} = 2^{100} - 1.
	  		\]
	  		
		\item[(ii)]  $\sum_{k=100}^\infty \frac{1}{2^k} = \sum_{k=100}^\infty \left( \frac{1}{2} \right)^k$.  
		
		This is a geometric series with $a=\frac{1}{2^{100}}$ and $r = \frac{1}{2}$.  
		So
			\[
			\sum_{k=100}^\infty \frac{1}{2^k} = \frac{\frac{1}{2^{100}}}{1-\frac{1}{2}} = \frac{1}{2^{99}}.
			\]
			
	Therefore, combining parts (i) and (ii) we have that
		\[
		\sum_{k=0}^{99} 2^k + \sum_{k=100}^\infty \frac{1}{2^k} = 2^{100} - 1 + \frac{1}{2^{99}}.
		\]
		\end{enumerate}
	\end{freeResponse}
	
	
	
	\item  $\sum_{k=0}^\infty (\cos(1))^k$
	\begin{freeResponse}
	
	\end{freeResponse}
	
	
	
	\item  $\sum_{k=4}^\infty \frac{5 \cdot 4^{k+3}}{7^{k-2}}$
	\begin{freeResponse}
	
	\end{freeResponse}
	
	
	
	\item  $\sum_{k=0}^\infty e^{5-2k}$
	\begin{freeResponse}
	
	\end{freeResponse}
	
	
	
	\item  $\sum_{k=0}^\infty \frac{e^k + (-7)^k}{5^k}$
	\begin{freeResponse}
	
	\end{freeResponse}
	
	
	
	\item  $ \sum_{k=0}^\infty \left[ \frac{5}{(k+1)(k+2)} + \left( - \frac{1}{2} \right)^k \right]$
	\begin{freeResponse}
	
	\end{freeResponse}
	
	
	
	\item  $\sum_{i=1}^\infty \left( \frac{1}{i} - \frac{1}{i+2} \right)$
	\begin{freeResponse}
	
	\end{freeResponse}
	
	\end{enumerate}
	
\end{problem}

\begin{instructorNotes}
Assign two per group.  
Most of these involve geometric series and one (part (b)) involves a finite geometric sum, whose ``trick'' is presented in the lecture.  

Students must identify the ``$r$'' and pay attention to both the indices and the exponents (for example, $7^{k+3} = 7^3 \cdot 7^k$).  
Encourage multiple methods (there are $3$ methods presented in the lecture) and make sure students clearly explain their reasoning on paper.  
\end{instructorNotes}







%problem 2
\begin{problem}
Convert the decimal $2.456\overline{314}$ to a fraction using geometric series.
	\begin{freeResponse}
		\begin{align*}
		2.456\overline{314}
		&= 2.456 + 0.000314 + 0.000000314 + \hdots  \\
		&= 2.456 + \frac{314}{1000^2} + \frac{314}{1000^3} + \hdots  \\
		&= 2.456 + \sum_{k=1}^\infty \left[ \frac{314}{1000} \cdot \left( \frac{1}{1000} \right)^k \right]  \\
		&= \frac{2456}{1000} + \frac{\frac{314}{1000^2}}{1 - \frac{1}{1000}}  \\
		&=  \frac{2456}{1000} + \frac{\frac{314}{1000^2}}{\frac{999}{1000}}  \\
		&= \frac{2456}{1000} + \frac{314}{999000}  \\
		&= \frac{2453544 + 314}{999000}  \\
		&= \frac{2453858}{999000} = \frac{1226929}{499500}.
		\end{align*}
	\end{freeResponse}

\end{problem}

\begin{instructorNotes}
This is a common type of problem in this section.  
Students have (most likely) never seen a problem with a non-repeating part to the decimal.  
This should probably be done as a whole class.
\end{instructorNotes}








%problem 3
\begin{problem}
Find all values of $x$ for which the series 
$$f(x) = \sum_{k=0}^\infty \frac{(x+3)^k}{2^k}$$ 
converges.
	\begin{freeResponse}
	First notice that
		\[
		f(x) = \sum_{k=0}^\infty \frac{(x+3)^k}{2^k} = \sum_{k=0}^\infty \left( \frac{x+3}{2} \right)^k
		\]
	and so this is a geometric series with $a=1$ and $r = \frac{x+3}{2}$.  
	So this series converges when
		\begin{align*}
		\biggr| &\frac{x+3}{2} \biggr| < 1  \\
		\Longleftrightarrow 	\quad	-1 < &\frac{x+3}{2} < 1  \\
		\Longleftrightarrow		\quad	-2 < &x+3 < 2  \\
		\Longleftrightarrow		\quad	-5 < &x < -1.
		\end{align*}
	\end{freeResponse}
		
\end{problem}

\begin{instructorNotes}
The purpose of this problem is to give the students a preview of the idea of an interval of convergence (to be covered in Chapter 10).  
Students need to be careful with absolute values and behavior at endpoints.  
This could be skipped (or presented as a ``take home and think about it'' question).  
\end{instructorNotes}




















	
	
	
	
	
	
	
	
	

	










								
				
				
	














\end{document} 


















