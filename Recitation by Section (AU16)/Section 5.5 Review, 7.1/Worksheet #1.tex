\documentclass[a4paper,12pt]{article}
\usepackage{fullpage}
 \usepackage{graphicx, amsmath, amsthm, wrapfig}
\begin{document}

\flushleft

\Large
\underline{Worksheet \#1: Review of Differentiation and Basic Integration Skills}

\normalsize

\vspace{3mm}

The following worksheet is designed to help review and/or sharpen your ability to differentiate and integrate functions encountered in a typical Calculus 1 course.  These problems are all reasonable to expect to see on the quiz this coming Friday (and each Friday thereafter).

\vspace{5mm}

\textbf{I. Differentiation Practice}

Differentiate the following functions.
\begin{align*}
a)  \, \, y &= (2x-7)^4  & b) \, \, y &=  e^{\frac{x}{4}} & c) \, \, y&= 7x^4-3\sqrt[5]{x} + \dfrac{2}{5x^2} \\
d) \, \, y&= \ln(2x+\cos x) & e) \, \, y&=2xe^{-x} & f) \, \, y&= \dfrac{\tan{(3x)}}{\sqrt{4-x}} \\
g) \, \, y&= \csc{\left(e^{4x}\right)} & h) \, \, y&= [\ln(4x^3-2x)]^3 & i) \, \, y&= e^{4 \sqrt{x}} \\
j) \, \, y&=4e^{x \sin x} & k) \, \, y&= 6x^9-\dfrac{1}{8x^4}+\dfrac{2}{\sqrt[3]{2x-1}} & l) \, \, y &=\dfrac{2}{(3x^2-1)^2} \\
\end{align*}

\textbf{II. Integration Practice}

Compute the following integrals.  If an integral cannot be algebraically reduced to one of the basic functions (powers of $x$, trig functions, exponentials, etc) that can be easily integrated, state so!  

\begin{align*}
a) & \int \left(3x^4-\sqrt[3]{x^2} + \dfrac{2}{\sqrt[7]{x}}\right)  \, dx  & b) & \int e^{x^2} \, dx & c) & \int^{\pi/6}_{0} 4 \sin(2x) \, dx  \\
d) & \int e^{-\frac{x}{3}} \, dx  & e) & \int \ln x \, dx & f) & \int \dfrac{4x^3-3x}{2x^2} \, dx  \\
g) & \int \left( \sec(4x) \tan(4x) + 3 \sec^2 \frac{x}{5} \right) \, dx  & h) & \int^4_1 (\sqrt{x}-1)^2 \, dx  & i) & \int_0^1 \sqrt{e^{3x}} \, dx  \\
j) &\int \cot^2 (3x) \sec^2 (3x) \, dx & k) & \int \cos \sqrt{x} \, dx & l) & \int \dfrac{2}{(3x)^2} \\
\end{align*}

\textbf{III. Miscellaneous}

The following questions help dispel common integration errors and allow for one to gain some insight as to why these incorrect methods fail.
\begin{enumerate}
\item
Consider the function $f(x)=e^{2x}$.  We know that $\dfrac{d}{dx} \left( e^{2x} \right) = 2 e^{2x}$ by the Chain Rule, and this lets us easily conclude that $\int e^{2x} \, dx = \frac{1}{2} e^{2x}$.  This could of course be verified by u-substitution (if you know/remember this technique), but can also be understood the following way:

The symbol $\int e^{2x} \, dx$ represents a function whose derivative is $e^{2x}$.  Since taking a derivative of $e^{2x}$ results in multiplying $e^{2x}$ by 2, when we antidifferentiate $e^{2x}$, we must multiply by $\frac{1}{2}$.  

You must be careful with this type of thought!  Indeed, it works only when the argument of the function (in this case, the expression in the exponent) is LINEAR\footnote{``linear in x" means the argument is of the form $ax+b$} in x!

\begin{itemize}
\item[a)] Calculate $\dfrac{d}{dx} \left(e^{x^2} \right) $.
\item[b)] Suppose a student tries to apply the above logic to compute $\int e^{x^2} \, dx$.  The student concludes that since $\dfrac{d}{dx} e^{x^2} \, dx = 2x e^{x^2} $, then: 
\begin{equation}
 \int e^{x^2} \, dx = \dfrac{1}{2x} e^{x^2} \label{bad1}
\end{equation}

Since you know that $\int e^{x^2} \, dx$ is a function whose derivative is $e^{x^2}$, prove this student wrong by differentiating his/her answer (i.e. the RHS of Eqn \ref{bad1}).  

\item[c)] What insight does this reveal as to why this students' answer is wrong?  Why can we think of antidifferentiating $e^{2x}$ differently than antidifferentiating $e^{x^2}$?
\end{itemize}

\item
Another student sees the following integral on an exam:  $$\int \dfrac{7x^6-3x^2}{4x^3} \, dx$$The student answers the question the following way:
\begin{align}
\int \dfrac{7x^6-3x^2}{4x^3} \, dx &=  \dfrac{\int \left(7x^6-3x^2\right) \, dx} {\int 4x^3 \, dx} \nonumber \\
&= \dfrac{x^7-x^3}{x^4} + C \label{bad2}
\end{align}

\begin{itemize}
\item[a)] By using Eqn. \ref{bad2} exactly as it is written above (i.e. WITHOUT simplifying it!), show that the derivative of the RHS of Eqn. \ref{bad2} is NOT equal to the expression in the original integrand.
\item[b)] What insight does this yield?  Why can one not simply just integrate the numerator and denominator of a fraction separately?
\item[c)] Compute the antiderivative of this function correctly.
\end{itemize}
\end{enumerate}

\end{document} 
