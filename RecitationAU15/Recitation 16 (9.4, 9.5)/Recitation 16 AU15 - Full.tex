\documentclass[]{ximera}
%handout:  for handout version with no solutions or instructor notes
%handout,instructornotes:  for instructor version with just problems and notes, no solutions
%noinstructornotes:  shows only problem and solutions

%% handout
%% space
%% newpage
%% numbers
%% nooutcomes

%I added the commands here so that I would't have to keep looking them up
%\newcommand{\RR}{\mathbb R}
%\renewcommand{\d}{\,d}
%\newcommand{\dd}[2][]{\frac{d #1}{d #2}}
%\renewcommand{\l}{\ell}
%\newcommand{\ddx}{\frac{d}{dx}}
%\everymath{\displaystyle}
%\newcommand{\dfn}{\textbf}
%\newcommand{\eval}[1]{\bigg[ #1 \bigg]}

%\begin{image}
%\includegraphics[trim= 170 420 250 180]{Figure1.pdf}
%\end{image}

%add a ``.'' below when used in a specific directory.
\input{../preamble.tex} %% we can turn off input when making a master document

\title{Recitation \#16: The Divergence, Integral, Ratio and Root Tests - Full}  

\begin{document}
\begin{abstract}		\end{abstract}
\maketitle












\section{Group work:}



%problem 1
\begin{problem}
For each of the following, answer {\bf True} or {\bf False}, and explain why.
	\begin{enumerate}
	
	\item  If $\sum_{n=0}^\infty a_n$ converges, then $\sum_{n=0}^\infty (a_n + 0.001)$ converges.
	
	\item  Since $\int_1^\infty x \sin(\pi x) \d x$ diverges then, by the Integral Test, $\sum_{n=0}^\infty n \sin(\pi n)$ diverges.
	
	\item  Since $\int_1^\infty \frac{1}{x^2} \d x = 1$ then, by the Integral Test, $\sum_{k=1}^\infty \frac{1}{k^2} = 1$.  
	
	\end{enumerate}
	
	\begin{freeResponse}
		\begin{enumerate}
		
		\item  {\bf False}
		
		Since $\sum_{n=0}^\infty a_n$ converges, we know that $\lim_{n \to \infty} a_n = 0$.  
		But then 
			\[
			\lim_{n \to \infty} (a_n + 0.0001) = 0.0001 \neq 0
			\]
		and so $\sum_{n=0}^\infty (a_n + 0.001)$ diverges by the Divergence Test.
		
		
		
		\item  {\bf False}
		
		The Integral Test only holds for positive, decreasing functions.  
		The function $f(x)= x \sin(\pi x)$ is not always positive, nor is it always decreasing.  
		So the Integral Test does not apply here.
		
		This problem is simpler than that though.  
		Since $\sin(\pi n) = 0$ for all integers $n$, we have that $\sum_{n=0}^\infty n \sin(\pi n) = 0$.
		
		
		
		\item  {\bf False}
		
		The Integral Test tells us that $\sum_{k=1}^\infty \frac{1}{k^2}$ converges, but it does {\bf not} give us the sum (this sum is actually $\frac{\pi^2}{6}$).  
		
		\end{enumerate}
	\end{freeResponse}
	
\end{problem}

\begin{instructorNotes}
For part (b) we need $f(x)$ to be a decreasing function for the Integral Test to (necessarily) hold.
All groups should do all of the parts.
\end{instructorNotes}







%problem 2
\begin{problem}
Assume $\sum_{k=0}^\infty a_k =L$ and $b_k = 8$ for all $k$. 
	\begin{enumerate}
	
	\item  What is $\lim_{k \to \infty} (a_k + b_k)$?
	
	\item  What is $\lim_{k \to \infty} \sum_{n=0}^k (a_n + b_n)$?
	
	\item  What is $\lim_{k \to \infty} \sum_{n=0}^k (a_{n+1} - a_n)$?
	
	\end{enumerate}
	
	\begin{freeResponse}
		\begin{enumerate}
	
		\item  Since $\sum_{k=0}^\infty a_k$ converges, we know that $\lim_{k \to \infty} a_k = 0$.  
		Therefore,
			\[
			\lim_{k \to \infty} (a_k + b_k) = 0 + 8 = \boxed{8}.
			\]
	
		\item  Since $\lim_{n \to \infty} (a_n + b_n) = 8$, the series $\sum_{n=0}^\infty (a_n + b_n)$ diverges by the Divergence Test.  
		But $\lim_{k \to \infty}  \sum_{n=0}^k (a_n + b_n) = \sum_{n=0}^\infty (a_n + b_n)$.  
		Thus
			\[
			\lim_{k \to \infty}  \sum_{n=0}^k (a_n + b_n) = \sum_{n=0}^\infty (a_n + b_n) = \boxed{\infty}.
			\]
	
		\item  Let $S_k = \sum_{n=0}^k (a_{n+1} - a_n)$ (and recall that $\{ S_k \}$ is the {\it sequence of partial sums}).
		Then
			\begin{align*}
			S_k &= \sum_{n=0}^k (a_{n+1} - a_n)  \\
			&= (a_1 - a_0) + (a_2 - a_1) + (a_3 - a_2) + \hdots + (a_k - a_{k-1}) + (a_{k+1} - a_k)  \\
			&= a_{k+1} - a_0.
			\end{align*}
		Thus,
			\[
			\lim_{k \to \infty} \sum_{n=0}^k (a_{n+1} - a_n) = \lim_{k \to \infty} S_k = \lim_{k \to \infty} a_{k+1} - a_0 = \boxed{-a_0}.
			\]
	
		\end{enumerate}
	\end{freeResponse}
		
\end{problem}

\begin{instructorNotes}
This question was adapted from midterm \#2 in Spring 2013.  
Students had difficulty distinguishing between a question dealing with sequences vs. a question dealing with series.
\end{instructorNotes}





%Root and Ratio Test Problem
%problem 1
\begin{problem}
Determine if the following series converge or diverge.
	\begin{enumerate}
	
	\item  $\sum_{n=1}^\infty \frac{(7n+1)^2 \cdot 2^n}{5^n}$
	
	\item  $\sum_{n=1}^\infty a_n$, where $a_{n+1} = \frac{2n+5}{3n-1} \cdot a_n$ and $a_1 = 1$.
	
	\item  $\sum_{n=0}^\infty \frac{n^2 + 2n + 1}{3n^2 +1}$
	
	\item  $\sum_{n=2}^\infty \frac{1}{n(\ln n)^2}$
	
	\item $\sum_{k=1}^{\infty} \frac{(k!)^3}{(3k)!}$
	
	\end{enumerate}
	
	\begin{freeResponse}
		\begin{enumerate}
	
		\item  \dfn{Ratio Test}
			\begin{align*}
			\lim_{n \to \infty} \frac{a_{n+1}}{a_n} 
			&= \lim_{n \to \infty} \left[ \frac{(7(n+1) + 1)^2 \cdot 2^{n+1}}{5^{n+1}} \cdot \frac{5^n}{(7n+1)^2 \cdot 2^n} \right]  \\
			&= \lim_{n \to \infty} \frac{(7n+8)^2 \cdot 2}{5 \cdot (7n+1)^2}  \\
			&= \frac{49 \cdot 2}{49 \cdot 5} = \frac{2}{5}.
			\end{align*}
		Thus, since $\lim_{n \to \infty} \frac{a_{n+1}}{a_n} < 1$, this series \boxed{converges}.  
		
		
	
		\item  \dfn{Ratio Test}
		
		Even though the terms in this series look a little weird, this is set up perfectly for the Ratio Test:
			\[
			\lim_{n \to \infty} \frac{a_{n+1}}{a_n} = \lim_{n \to \infty} \frac{2n+5}{3n-1} = \frac{2}{3}.
			\]
		Thus, since $\lim_{n \to \infty} \frac{a_{n+1}}{a_n} < 1$, this series \boxed{converges}.  
		
		
	
		\item  \dfn{Divergence Test}
		
		Notice that
			\[
			\lim_{n \to \infty} a_n = \lim_{n \to \infty} \frac{n^2 + 2n + 1}{3n^2 +1} = \frac{1}{3}.
			\]
		Therefore, since $\lim_{n \to \infty} a_n \neq 0$, by the Divergence Test this series \boxed{diverges}.
		
		
	
		\item  \dfn{Integral Test}
		
		First, notice that $f(x) = \frac{1}{x (\ln x)^2}$ is a decreasing and positive function on $[2,\infty)$.
		Then
			\begin{align*}
			\int_2^\infty f(x) \d x 
			&= \int_2^\infty \frac{1}{x (\ln x)^2} \d x  \\
			&= \lim_{b \to \infty} \int_2^b \frac{1}{x (\ln x)^2} \d x  \\
			&= \lim_{b \to \infty} \int_{\ln 2}^{\ln b} u^{-2} \d u  \quad  {\color{red} u = \ln x, \d u = \frac{1}{x} \d x}  \\
			&= \lim_{b \to \infty} \eval{\frac{-1}{u}}_{\ln 2}^{\ln b}  \\
			&= \lim_{b \to \infty} \left( \frac{-1}{\ln b} + \frac{1}{\ln 2} \right)  \\
			&= 0 + \frac{1}{\ln 2} = \frac{1}{\ln 2}.
			\end{align*}
		Therefore, since the above integral converges, the series $\sum_{n=2}^\infty \frac{1}{n(\ln n)^2}$ \boxed{converges} by the Integral Test.
	

		\item \dfn{Ratio Test}
		\begin{align*}
			\lim_{k \to \infty} \frac{a_{k+1}}{a_k} 
			&= \lim_{k \to \infty} \left[ \frac{( (k+1)!)^3}{(3(k+1))!}  \cdot \frac{(3k)!}{(k!)^3} \right]  \\
			&= \lim_{k \to \infty} \frac{ (k+1)^3 (k!)^3}{(3k+3)(3k+2)(3k+1) \cdot (3k)!} \cdot \frac{(3k)!}{(k!)^3} \\
			&= \lim_{k \to \infty} \frac{(k+1)^3}{(3k+3)(3k+2)(3k+1)}  \\
			&= \frac{1}{3 \cdot 3 \cdot 3}= \frac{1}{27}.
			\end{align*}
		Thus, since $\lim_{k \to \infty} \frac{a_{k+1}}{a_k} < 1$, this series \boxed{converges}. 
		
		\end{enumerate}
	\end{freeResponse}
	
\end{problem}

\begin{instructorNotes}
Let the students experiment with what tests to use.  
Perhaps give two problems per group.
\end{instructorNotes}







%Root and Ratio Test Problem
%problem 2
\begin{problem}
How many terms are needed to estimate $\sum_{k=1}^\infty \frac{1}{k^2+1}$ to within $10^{-4}$?
What is the estimate for the sum of the series?
	\begin{freeResponse}
	Let $f(x) = \frac{1}{x^2 + 1}$.  
	Note that $f(x)$ is continuous, decreasing, and positive for $x \geq 1$, and $a_k = f(k)$.  
	
	If $S = \sum_{k=1}^\infty \frac{1}{k^2 + 1}$ is the actual value of the sum, and $S_n = \sum_{k=1}^n \frac{1}{k^2 + 1}$ is the $n^{th}$ partisl sum, then the remainder $R_n = S - S_n$ can be bounded above by
		\[
		R_n < \int_n^\infty f(x) \d x.
		\]
	So we need to find $n$ so that
		\begin{equation}\label{inequality}
		\int_n^\infty \frac{1}{x^2 + 1} \d x < 10^{-4}.
		\end{equation}
	We compute
		\begin{align*}
		\int_n^\infty \frac{1}{x^2 + 1} \d x
		&= \lim_{b \to \infty} \int_n^b \frac{1}{x^2+1} \d x  \\
		&= \lim_{b \to \infty} \eval{ \arctan(x) }_n^b  \\
		&= \lim_{b \to \infty} \left( \arctan(b) - \arctan(n) \right)  \\
		&= \frac{\pi}{2} - \arctan(n).
		\end{align*}
	Plugging into equation \eqref{inequality} we see that we want
		\begin{align*}
		&\frac{\pi}{2} - \arctan(n) < 10^{-4}  \\
		\Longrightarrow 	\quad	&\arctan(n) > \frac{\pi}{2} - 10^{-4}  \\
		\Longrightarrow 	\quad	&n > \tan \left( \frac{\pi}{2} - 10^{-4} \right) \approx 9999.999967 \approx 10,000.
		\end{align*}
		
	Therefore, $10,000$ terms will estimate $\sum_{k=1}^\infty \frac{1}{k^2+1}$ to within $10^{-4}$.  
	Using a computer, we can also compute
		\[
		\sum_{k=1}^{10,000} \frac{1}{k^2+1} \approx 1.07657.
		\]
	\end{freeResponse}
		
\end{problem}

\begin{instructorNotes}
This should be a straightforward calculation, but calculators may be needed.
\end{instructorNotes}
















	
	
	
	
	
	
	
	
	

	










								
				
				
	














\end{document} 


















