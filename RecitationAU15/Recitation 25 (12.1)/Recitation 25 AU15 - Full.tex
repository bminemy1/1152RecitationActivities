\documentclass[]{ximera}
%handout:  for handout version with no solutions or instructor notes
%handout,instructornotes:  for instructor version with just problems and notes, no solutions
%noinstructornotes:  shows only problem and solutions

%% handout
%% space
%% newpage
%% numbers
%% nooutcomes

%I added the commands here so that I would't have to keep looking them up
%\newcommand{\RR}{\mathbb R}
%\renewcommand{\d}{\,d}
%\newcommand{\dd}[2][]{\frac{d #1}{d #2}}
%\renewcommand{\l}{\ell}
%\newcommand{\ddx}{\frac{d}{dx}}
%\everymath{\displaystyle}
%\newcommand{\dfn}{\textbf}
%\newcommand{\eval}[1]{\bigg[ #1 \bigg]}

%\begin{image}
%\includegraphics[trim= 170 420 250 180]{Figure1.pdf}
%\end{image}

%add a ``.'' below when used in a specific directory.
\input{../preamble.tex} %% we can turn off input when making a master document

\title{Recitation \#25: Vectors in the plane - Full}  

\begin{document}
\begin{abstract}		\end{abstract}
\maketitle




\section{Warm up:}
\begin{enumerate}
\item What is the difference between the notations $\hat{\imath}$, $\hat{\jmath}$, $\hat{u}$ and $\mathbf{i}, \mathbf{j}, \mathbf{u}$?
\item Sketch the vectors $\mathbf{u}=\langle 1, -1 \rangle$ and $\mathbf{v}=\langle 2, 0 \rangle$. Now using your sketch of these vectors, sketch $\mathbf{u}-2\mathbf{v}$.
\end{enumerate}
	\begin{freeResponse}
	\begin{enumerate}
	\item There is no difference. The ``hat" versions are typically written by hand, while the ``bold" versions are typically in print. 
	\item To add vectors, we put the tail of the second vector on the head of the first. 


\includegraphics[scale=0.5]{Figure12-1-1}



	\end{enumerate}
	\end{freeResponse}
	
\begin{instructorNotes}
Reminds students that there are different equivalent vector notations. Also, students should understand vector addition geometrically.
\end{instructorNotes}








\section{Group work:}











%problem1
\begin{problem}
Suppose that $\mathbf{u}=\langle5, -1\rangle$ and $\mathbf{v}=\langle 2, 3\rangle$. Find the following quantities:
\begin{enumerate}
\item $-\mathbf{v}$
\item $3\mathbf{u}-4\mathbf{v}$
\item $|\mathbf{u}|$
\item $|\mathbf{u}-2\mathbf{v}|$
\end{enumerate}
	\begin{freeResponse}
	\begin{enumerate}
	\item $-\mathbf{v}= \langle -2, -3 \rangle$
	\item $3\mathbf{u}-4\mathbf{v} = \langle 15, -4 \rangle - \langle 8, 12 \rangle = \langle 7, -16 \rangle$.
	\item $|\mathbf{u}|= \sqrt{5^2+(-1)^2} = \sqrt{26}.$
	\item $|\mathbf{u}-2\mathbf{v}|=|\langle 1, -7\rangle| = \sqrt{1^2+(-7)^2}=\sqrt{50}=5\sqrt{2}.$
	\end{enumerate}
	\end{freeResponse}
		
\end{problem}

\begin{instructorNotes}
Simple vector operations
\end{instructorNotes}









%problem2
\begin{problem}
Suppose that $\mathbf{u}=3 \mathbf{i} - 4 \mathbf{j}$. Find the following:
\begin{enumerate}
\item A unit vector in the same direction of $\mathbf{u}$. 
\item All unit vectors parallel to $\mathbf{u}$. (How does differ from part (a)?)
\item Two vector parallel to $\mathbf{u}$ with length 10.
\end{enumerate}
	\begin{freeResponse}
	\begin{enumerate}
	\item $|\mathbf{u}|= \sqrt{3^2+(-4)^2} = 5$. A unit vector in the same direction is $\frac{\mathbf{u}}{|\mathbf{u}|}= \langle \frac{3}{5}, \frac{-4}{5} \rangle.$
	\item Parallel unit vectors are $\pm \frac{\mathbf{u}}{|\mathbf{u}|}$, which are $\langle \frac{3}{5}, \frac{-4}{5} \rangle$ and $\langle \frac{-3}{5}, \frac{4}{5} \rangle$. Note that parallel vectors include vectors in the opposite direction. 
	\item Since $\mathbf{u}$ has length 5, two parallel vectors of length 10 are $\pm 2\mathbf{u}$, which are $\langle 6, -8 \rangle$ and $\langle -6, 8 \rangle$. 
	\end{enumerate}
	\end{freeResponse}

\end{problem}

\begin{instructorNotes}
This question gauges whether they know the difference between vectors in the same direction and parallel vectors.
\end{instructorNotes}



%problem 3
\begin{problem}
Assume that $\vec{u} = \frac{1}{2} \hat{\imath} + \frac{\sqrt{3}}{2} \hat{\jmath}$ and $\vec{v} = \frac{\sqrt{3}}{2} \hat{\imath} - \frac{1}{2} \hat{\jmath}$.
	\begin{enumerate}
	\item  Show that $\vec{u}$ and $\vec{v}$ are unit vectors.
	\item  Write $\hat{\imath}$ as $a_1 \vec{u} + b_1 \vec{v}$ for some real numbers $a_1$ and $b_1$.  
	\item  Write $\hat{\jmath}$ as $a_2 \vec{u} + b_2 \vec{v}$ for some real numbers $a_2$ and $b_2$. 
	\end{enumerate}
	
	\begin{freeResponse}
	\begin{enumerate}
	\item  First, to show that both vectors are unit vectors, we compute their magnitudes that they are equal to one.
		\begin{align*}
		&| \vec{u} | = \sqrt{ \left( \frac{1}{2} \right)^2 + \left( \frac{\sqrt{3}}{2} \right)^2} = \sqrt{ \frac{1}{4} + \frac{3}{4}} = 1  \\
		&| \vec{v}| = \sqrt{ \left( \frac{\sqrt{3}}{2} \right)^2 + \left( - \frac{1}{2} \right)^2} = \sqrt{\frac{3}{4}+\frac{1}{4}} = 1.
		\end{align*}
	
	
	
	\item  
		\begin{align}
		\hat{\imath} &= a_1 \vec{u} + b_1 \vec{v}  \nonumber \\
		&= a_1 \left( \frac{1}{2} \hat{\imath} + \frac{\sqrt{3}}{2} \hat{\jmath} \right) + b_1 \left( \frac{\sqrt{3}}{2} \hat{\imath} - \frac{1}{2} \hat{\jmath} \right)  \nonumber \\
		&= \left( \frac{1}{2} a_1 + \frac{\sqrt{3}}{2} b_1 \right) \hat{\imath} + \left( \frac{\sqrt{3}}{2} a_1 - \frac{1}{2} b_1 \right) \hat{\jmath}.  \label{equation 1}
		\end{align}
	Therefore, we must have that
		\[
		\frac{1}{2} a_1 + \frac{\sqrt{3}}{2} b_1 = 1 	\qquad	\text{and} 	\qquad	\frac{\sqrt{3}}{2} a_1 - \frac{1}{2} b_1 = 0.
		\]
	Solving the right-hand equation for $b_1$ we have that
		\[
		b_1 = \sqrt{3} a_1.
		\]
	Plugging this into the left-hand equation gives
		\begin{align*}
		&1 = \frac{1}{2} a_1 + \frac{\sqrt{3}}{2} \cdot \sqrt{3} a_1  \\
		\Longrightarrow \qquad &1 = \frac{1}{2} a_1 + \frac{3}{2} a_1 = 2 a_1  \\
		\Longrightarrow \qquad &a_1 = \frac{1}{2}.
		\end{align*}
	So $b_1 = \frac{\sqrt{3}}{2}$, and therefore
		\[
		\boxed{\hat{\imath} = \frac{1}{2} \vec{u} + \frac{\sqrt{3}}{2} \vec{v}}.
		\]
	
	
	
	\item  In the same manner as in equation \eqref{equation 1} we have that
		\[
		\hat{\jmath} = \left( \frac{1}{2} a_2 + \frac{\sqrt{3}}{2} b_2 \right) \hat{\imath} + \left( \frac{\sqrt{3}}{2} a_2 - \frac{1}{2} b_2 \right) \hat{\jmath}
		\]
	and so
		\[
		\frac{1}{2} a_2 + \frac{\sqrt{3}}{2} b_2 = 0 	\qquad	\text{and} 	\qquad	\frac{\sqrt{3}}{2} a_2 - \frac{1}{2} b_2 = 1.
		\]
	Now, solving the left-hand equation for $a_2$ gives
		\[
		a_2 = - \sqrt{3} b_2.
		\]
	Then plugging into the right-hand equation yields
		\begin{align*}
		&1 = \frac{\sqrt{3}}{2} \cdot (- \sqrt{3} b_2 ) - \frac{1}{2} b_2  \\
		\Longrightarrow \qquad &1 = - \frac{3}{2} b_2 - \frac{1}{2} b_2 = -2 b_2  \\
		\Longrightarrow \qquad &b_2 = - \frac{1}{2}.
		\end{align*}
	So $a_2 = \frac{\sqrt{3}}{2}$, and therefore
		\[
		\boxed{\hat{\jmath} = \frac{\sqrt{3}}{2} \vec{u} - \frac{1}{2} \vec{v}   }
		\]
	
	\end{enumerate}
	\end{freeResponse}
	
\end{problem}

\begin{instructorNotes}
Students tend to have issues writing a vector as a linear combination of other vectors.
\end{instructorNotes}













	
	
	
	
	
	
	
	
	

	










								
				
				
	














\end{document} 


















