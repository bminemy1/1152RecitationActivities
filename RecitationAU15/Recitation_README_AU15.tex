\documentclass{article}

\usepackage{amsfonts, amssymb, amsmath}









\begin{document}

\large

\begin{itemize}
\item In the later recitation folders, there are four pdf files. One of these is rather unnecessary. I suggest doing only the recitation handout, solutions and "Full" ones. The "instructor notes" is unnecessary in light of the "Full" which includes all the information. The handout and solutions are for the students and are put up on carmen. The "Full" pdfs are for the instructor. I started doing the "Full" pdfs sometime during the semester. These "Full" have the questions, solutions and instructor notes. 
\item Do more warm-ups on the earlier recitation handouts. Some of the warm-ups are not on the master section files just the recitation handouts for this semester. I made sure to add the later ones, but I don't remember when I started.
\item Recitation 9 and 10: Both of these days were much too hard for the students. 
\item Recitation 11: Do a warm-up for a simple partial fraction decomposition.
\item Recitation 12: Problem 1: Can't have two solutions here.
\item Recitation 13: Problem 2 wording is weird. Also we can make problem 3 better by actually including more than just 1. Also perhaps we should include a graph of a direction field that depends on the horizontal variable. Problem 4: Include an example with a general solution like $y=4\pm \sqrt{x^3+C}$. 
\item Recitation 14: Problem 3b: Can not combine adjacent terms like this. e.g. $+1-1+1-1+1-1+\ldots$
\item Recitation 15: Warm-up. Include a limit of a sequence like $(1+b/n)^n$. Perhaps there is a better sequence to use in problem 2.
\item Recitation 16: In the future, it would be best to do question 3 (directly applying tests) rather than question 1 and 2.
\item Recitation 18: First, the warm-up is too difficult. Perhaps do an easy limit comparison test question on there. On Prob 2, put a root of a rational function here for limit comparison test. For 4b, perhaps try an easier summand. 
\item Recitation 19: For section 10.1, make more questions finding $p_k(x)$ and dealing with the remainder. Perhaps a question showing the graphs of $f(x)$, $p_1(x)$, $p_2(x)$ and so on.  
\item Recitation 20: Students needed a refresher for series center and radius of convergence. Also I think 1c is a bad problem. It is not a power series, it does not need to have an interval of convergence. Consider $\sum_k 2^k x^k - (x-1)^k$. Via the method here, you would conclude that the interval of convergence is (0,1/2). Plug in $x=-1$. Maybe do a problem such as $\sum_k a_k (x-2)^k$ has an interval of convergence [-2,6). Find all $x$-values where the series $\sum_k a_k (3x^2-2)^k$ converges. Or even something trickier like having the initial interval of convergence [1,3).  In the solution to Problem 3, suggest that there are muliple other power series representations that have other centers and perhaps give one. In problem 2, in (a) switch $3^k$ with $4^k$; in (b), instead of $x^k$, make it $x^{2k}/3^k$. 
\item Recitation 21: Warm-up: (b) is too hard. Perhaps ask what ${-3 \choose 0}, {-3 \choose 1}, \ldots, {-3\choose 4}$ are. Also, add find the power series for $\frac{1}{(1+x)^4}$ in the warm-up. On problem 1, change (b) to $\ln (1-2x^2)$. On 2, add a (b) $\sin(3x^2)$. On 3, add (c) $\sin (\pi)$ and (d) $e^e$. 
\item Recitation 22: Warm-up: Add T/F To approximate $\pi/6$, one could substitute $x=1/\sqrt{3}$ into the Maclaurin series for $\tan^{-1}(x)$. Make problem 3 have an "a" and a "b". (a) $\sum_{k=0}^{\infty} k x^k$. Make problem 4 easier such as $y'-xy=1$ $y(0)=0$. 
\item Recitation 23: Warm-up: I suggest switching $x$ and $y$. That way, they realize that when $x(t)$ is increasing you move to the right and when $x(t)$ is decreasing you move to the left. Problem 1: Restrict to $t \geq 1$. Make problem 6 be the point $(-\sqrt{3}, 1)$. Students have issues with finding angles in the 2nd and 3rd quadrants.
\item Recitation 24: Problem 1: It is too difficult to find the $\theta$ values where the tangent lines are vertical. Problem 2: I suggest starting with an $(a)$ Inside the curve $r=3-2\cos \theta$ and outside the curve $r=2$ for each group. Then break up the other two among the groups. 
\item Recitation 25: Perhaps have better vectors in problem 3. Don't necessarily need to be unit vectors. Maybe a vector addition/subtraction question that only gives them the graphs of the vectors. 
\item Recitation 26: Problem 2: Use the vector $\langle \sqrt{3}, 1 \rangle$ rather than $\langle 3, 4 \rangle$. Problem 3b: On solution, the picture is wrong. The scalar projection is negative, so $\text{proj}_{\vec{v}}(\vec{u})$ heads in the opposite direction of $\vec{v}$. Also make the numbers for this projection nicer. A problem showing two vectors are orthogonal would be good.
\item Recitation 27: Problem 2: Change the second vector to $\langle 1, 0, 2 \rangle$. In this case, the numbers work better. 
\item Recitation 29: Problem 1: Should this point (1,-2,3) be a intersection point for the other two lines? Better numbers in 2. Variables are hard to see in the graphs for Problem 4. 
\end{itemize}


\end{document}
