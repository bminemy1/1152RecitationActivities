\documentclass[]{ximera}
%handout:  for handout version with no solutions or instructor notes
%handout,instructornotes:  for instructor version with just problems and notes, no solutions
%noinstructornotes:  shows only problem and solutions

%% handout
%% space
%% newpage
%% numbers
%% nooutcomes

%I added the commands here so that I would't have to keep looking them up
%\newcommand{\RR}{\mathbb R}
%\renewcommand{\d}{\,d}
%\newcommand{\dd}[2][]{\frac{d #1}{d #2}}
%\renewcommand{\l}{\ell}
%\newcommand{\ddx}{\frac{d}{dx}}
%\everymath{\displaystyle}
%\newcommand{\dfn}{\textbf}
%\newcommand{\eval}[1]{\bigg[ #1 \bigg]}

%\begin{image}
%\includegraphics[trim= 170 420 250 180]{Figure1.pdf}
%\end{image}

%add a ``.'' below when used in a specific directory.
\input{../preamble.tex} %% we can turn off input when making a master document

\title{Recitation 23: Parametric equations \& Polar coordinates - Full}  

\begin{document}
\begin{abstract}		\end{abstract}
\maketitle




\section{Warm up:}
Describe the motion given by $x=8$, $y=7 \sin (t)$ for all $t$.
	\begin{freeResponse}
	The parametric curve keeps oscillating up and down the vertical line segment between the points $(8,-7)$ and $(8,7)$.
	This is because $x=8$ is fixed, and $y = 7 \sin(t)$ oscillates between $-7$ and $7$ as $t$ varies.
	\end{freeResponse}
	
\begin{instructorNotes}
One variable is held constant while the other oscillates up and down.
\end{instructorNotes}







\section{Group work:}



%problem 1
\begin{problem}
Try to figure out the shape of the following curve and then eliminate the parameter and check your intuition.
	\[
	x = \ln t - 1 \qquad y = (\ln t)^2
	\]
	\begin{freeResponse}
	First, note that these two functions are both only defined when $0 < t < \infty$.  
	Also, we see that $\ln t = x + 1$, and so 
		\[
		y = (\ln t)^2 = (x+1)^2.
		\]
	So this graph is a parabola that opens up and has vertex $(-1,0)$.  
	
	\begin{image}
	\includegraphics[trim= 170 270 170 310]{Figure11-1-1.pdf}
	\end{image}
	\end{freeResponse}
	
\end{problem}

\begin{instructorNotes}
It is not hard to see that $y = (x+1)^2$.  
So this is a parabola.  
\end{instructorNotes}







%problem 2
\begin{problem}
Find parametric equations for the path of a particle moving around the circle
	\[
	(x-3)^2 + (y+7)^2 = 4
	\]
	
	\begin{enumerate}
	\item  one time around clockwise starting at $(5,-7)$.  
	\item  three times around counterclockwise starting at $(5,-7)$.  
	\item  halfway around clockwise starting at $(-1,-7)$.  
	\end{enumerate}
	
	\begin{freeResponse}
	First, notice that this is the equation of the circle with radius $2$ centered at $(3,-7)$.  
	\begin{enumerate}
	\item  The point $(5,-7)$ is the ``right-most'' point on the circle.  
	In order to parameterize the circle one time around going {\it counter-clockwise} we use the parameterization
		\[
		x = 3 + 2 \cos t 	\qquad 	y = -7 + 2 \sin t 	\qquad	0 \leq t < 2 \pi.
		\]
	In order to traverse the circle one time {\it clockwise}, we just negate the $t$ in the above parameterization.  
	So we get
		\[
		\boxed{x = 3 + 2 \cos(-t) 	\qquad	y = -7 + 2 \sin(-t) 	\qquad	0 \leq t < 2 \pi}.
		\]
	{\it Remark:  Since $\cos$ is an even function and $\sin$ is an odd function, this solution is equivalent to
		\[
		x = 3 + 2 \cos t 	\qquad	y = -7 - 2 \sin t 	\qquad	0 \leq t < 2 \pi
		\]
	}
	
	
	
	\item  To traverse the circle three times counter-clockwise, we just triple the domain of $t$ from the parameterization above.  
	So we have that
		\[
		\boxed{x = 3 + 2 \cos t 	\qquad 	y = -7 + 2 \sin t 	\qquad	0 \leq t < 6 \pi}.
		\]
	
	
	
	\item  Note that this problem starts at $(-1,-7)$, not $(5,-7)$.  
	So this parameterization begins at the ``left-most'' point of the circle.  
	Therefore, this is just the ``second half'' of the answer to part (a).  
	So a parameterization for this problem is
		\[
		\boxed{x = 3 + 2 \cos(-t) 	\qquad	y = -7 + 2 \sin(-t) 	\qquad	\pi \leq t < 2 \pi}.
		\]
	
	\end{enumerate}
	\end{freeResponse}
		
\end{problem}

\begin{instructorNotes}
Practice with the formula for parameterizing a circle.  
It is useful to point out that the same curve can have many different parameterizations.
\end{instructorNotes}







%problem 3
\begin{problem}
Find the intersection point(s) of the lines
	\begin{equation}\label{line 1}
	x=-6 + 9t, 	\qquad 	y = 3-2t
	\end{equation}
and
	\begin{equation}\label{line 2}
	x=3+t, 	\qquad	y=-4-2t.
	\end{equation}
Do they intersect at the same time?
	\begin{freeResponse}
	Line \ref{line 1} is the line with slope $- \frac{2}{9}$ passing through $(-6,3)$.  
	So it has equation
		\begin{align*}
		&y - 3 = - \frac{2}{9} (x-(-6))  \\
		\Longrightarrow 	\qquad 	&y = - \frac{2}{9} x - \frac{4}{3} + 3 = - \frac{2}{9} x + \frac{5}{3}.
		\end{align*}
		
	Line \ref{line 2} is the line with slope $- 2$ passing through $(3,-4)$.  
	So it has equation
		\begin{align*}
		&y + 4 = -2 (x- 3)  \\
		\Longrightarrow 	\qquad 	&y = - 2x + 2.
		\end{align*}
		
	Since these lines have different slopes, they intersect in a single point.  
	To find this point, we set the equations equal to each other:
		\begin{align*}
		&- \frac{2}{9} x + \frac{5}{3} = -2x+2  \\
		\Longrightarrow 	\qquad 	&\frac{16}{9} x = \frac{1}{3}  \\
		\Longrightarrow 	\qquad 	&x = \frac{3}{16}  \\
		\Longrightarrow 	\qquad 	&y = -2 \left( \frac{3}{16} \right) + 2 = \frac{13}{8}.
		\end{align*}
	Therefore, the intersection point is
		\[
		\boxed{ \left( \frac{3}{16}, \frac{13}{8} \right)}.
		\]
		
	To see if they intersect in the same point, let us first find the $t$ value which makes the $x$-coordinate of line \ref{line 1} equal to $\frac{3}{16}$.  
		\begin{align*}
		&\frac{3}{16} = -6 + 9t  \\
		\Longrightarrow 	\qquad 	&t = \frac{1}{9} \left( \frac{3}{16} + 6 \right) = \frac{11}{16}.
		\end{align*}
	So now, we plug this into the equation for $x(t)$ for line \ref{line 2} and see if we get $\frac{3}{16}$.
		\begin{align*}
		x \left( \frac{11}{16} \right) &= 3 + \frac{11}{16}  \\
		&= \frac{59}{16} \neq \frac{3}{16}.
		\end{align*}
	Therefore, these lines do {\bf not} intersect at the same time.
	\end{freeResponse}

\end{problem}

\begin{instructorNotes}
The key point here is that intersections can occur at different ``times''.
\end{instructorNotes}







%problem 4
\begin{problem}
Consider the curve defined by the parameterization $x=t^2$, $y= t^3 - 3t$.  
Show that this curve has two tangent lines at $(3,0)$, and find the equations of the tangent lines there.
	\begin{freeResponse}
	First, recall that
		\[
		\frac{\d y}{\d x} = \frac{\frac{\d y}{\d t}}{\frac{\d x}{\d t}}.
		\]
	Then, since $\frac{\d x}{\d t} = 2t$ and $\frac{\d y}{\d t} = 3t^2 - 3$, we have that
		\[
		\frac{\d y}{\d x} = \frac{3t^2 - 3}{2t}.
		\]
	Now,
		\begin{align*}
		&x(t) = 3  \\
		\Longleftrightarrow 	\qquad	&t^2 = 3  \\
		\Longleftrightarrow 	\qquad	&t = \pm \sqrt{3}.
		\end{align*}
	Also, notice that both $y(\sqrt{3}) = 0 = y(- \sqrt{3})$.  
	So the given parametric curve intersects the point $(3,0)$ at two times, when $t = \pm \sqrt{3}$.  
	At these two times, the tangent lines have slopes
		\[
		\frac{\d y}{\d x} \biggr|_{t = \sqrt{3}} = \frac{(3)(3) - 3}{2 \sqrt{3}} = \sqrt{3}	\qquad	\text{and}	\qquad	\frac{\d y}{\d x} \biggr|_{t = - \sqrt{3}} = \frac{(3)(3) - 3}{-2 \sqrt{3}} = -\sqrt{3}.
		\]
	So the equations of the tangent lines are
		\begin{align*}
		&\boxed{y = \sqrt{3} (x-3)}  \\
		&\boxed{y = -\sqrt{3}(x-3)}.
		\end{align*}
	\end{freeResponse}

\end{problem}

\begin{instructorNotes}
The main idea here is that a parametric curve can have several tangent lines at the same point (in $(x,y)$ coordinates).
\end{instructorNotes}





%problem 5
\begin{problem}
Plot the following (polar) points in the $xy$-plane and then rewrite them as rectangular coordinates.
	\begin{multicols}{4}
	\begin{enumerate}
	\item  $\left( 3, \frac{5 \pi}{4}   \right) $
	\item  $\left( 3, - \frac{5 \pi}{4}   \right) $
	\item  $\left( -3, \frac{5 \pi}{4}   \right) $
	\item  $\left( -3, - \frac{5 \pi}{4}   \right) $
	\end{enumerate}
	\end{multicols}
	
	\begin{freeResponse}
	\begin{enumerate}
	\item  
		\[
		x = 3 \cos \left( \frac{5 \pi}{4} \right) = - \frac{3 \sqrt{2}}{2}
		\]
		\[
		y = 3 \sin \left( \frac{5 \pi}{4} \right) = - \frac{3 \sqrt{2}}{2}
		\]
		
		\begin{image}
		\includegraphics[trim= 170 320 170 320, scale=0.7]{Figure11-2-2.pdf}
		\end{image}
	
	\item  
		\[
		x = 3 \cos \left( - \frac{5 \pi}{4} \right) = - \frac{3 \sqrt{2}}{2}
		\]
		\[
		y = 3 \sin \left( - \frac{5 \pi}{4} \right) = \frac{3 \sqrt{2}}{2}
		\]
		
		\begin{image}
		\includegraphics[trim= 170 320 170 320, scale=0.7]{Figure11-2-3.pdf}
		\end{image}
	
	\item  
		\[
		x = -3 \cos \left( \frac{5 \pi}{4} \right) = \frac{3 \sqrt{2}}{2}
		\]
		\[
		y = -3 \sin \left( \frac{5 \pi}{4} \right) = \frac{3 \sqrt{2}}{2}
		\]
		
		\begin{image}
		\includegraphics[trim= 170 310 170 310, scale=0.7]{Figure11-2-4.pdf}
		\end{image}
	
	\item  
		\[
		x = -3 \cos \left( - \frac{5 \pi}{4} \right) = \frac{3 \sqrt{2}}{2}
		\]
		\[
		y = -3 \sin \left( -\frac{5 \pi}{4} \right) = - \frac{3 \sqrt{2}}{2}
		\]
		
		\begin{image}
		\includegraphics[trim= 170 310 170 300, scale=0.7]{Figure11-2-5.pdf}
		\end{image}
	
	\end{enumerate}
	\end{freeResponse}
	
\end{problem}

\begin{instructorNotes}
Some students will be familiar with polar coordinates while others will have never worked with them before.  
So it is very important to establish how these coordinates are measured.

Students sometimes have a difficult time dealing with negative values for $r$ as well as with working with these while graphing.  
Both this and the next problem force the students to deal with these issues.  
\end{instructorNotes}







%problem 6
\begin{problem}
Rewrite the rectangular point $(3,5)$ in polar coordinates in three different ways.
	\begin{freeResponse}
	\begin{enumerate}
	\item[i.]
		\[
		r = \sqrt{3^2 + 5^2} = \sqrt{34}	\qquad	\theta = \arctan \left( \frac{5}{3} \right)
		\]
		
		\begin{image}
		\includegraphics[trim= 170 310 170 300, scale=0.6]{Figure11-2-6.pdf}
		\end{image}
	
	\item[ii.]
		\[
		r = \sqrt{34} 	\qquad	\theta = \arctan \left( \frac{5}{3} \right) - 2 \pi
		\]
		
		\begin{image}
		\includegraphics[trim= 170 310 170 290, scale=0.6]{Figure11-2-7.pdf}
		\end{image}
	
	\item[iii.]
		\[
		r = - \sqrt{34}	\qquad	\theta = \arctan \left( \frac{5}{3} \right) + \pi
		\]
		
		\begin{image}
		\includegraphics[trim= 170 290 170 260, scale=0.6]{Figure11-2-8.pdf}
		\end{image}
	
	\end{enumerate}
	\end{freeResponse}
		
\end{problem}

\begin{instructorNotes}
See problem 5.
\end{instructorNotes}







%problem 7
\begin{problem}
The graph of the curve $r = 4 \sin \theta$ is a circle. Use the graph below to sketch this circle. 
Can you verify this algebraically?  
What is the period of the polar curve?  
Is $0 \leq \theta \leq 2 \pi$ necessary to complete the graph?
	
	\begin{image}
	\includegraphics[trim= 170 290 170 280, scale=1]{Figure11-2-1.pdf}
	\end{image}

	\begin{freeResponse}
	Graphing this equation in the picture below, we see that this is a circle with radius $2$ and center $(0,2)$.  
	
	\begin{image}
	\includegraphics[trim= 170 310 170 290, scale=0.8]{Figure11-2-9.pdf}
	\end{image}
	
	To verify this algebraically,
		\begin{align*}
		4 = 4 \sin \theta \qquad
		&\Longrightarrow 	\qquad	r^2 = 4r \sin \theta  \\
		&\Longrightarrow 	\qquad	x^2 + y^2 = 4y  \\
		&\Longrightarrow 	\qquad	x^2 + y^2 - 4y = 0  \\
		&\Longrightarrow 	\qquad	x^2 + y^2 - 4y + 4 = 4  \\
		&\Longrightarrow 	\qquad	x^2 + (y-2)^2 = 2^2.
		\end{align*}
		
	To find the period of the polar curve, we convert both $x$ and $y$ into parametric equations with parameter $\theta$.  
		\begin{align*}
		&x = r \cos \theta = 4 \sin \theta \cos \theta = 2 \sin (2 \theta)  \\
		&y = r \sin \theta = 4 \sin^2 \theta = 4 \cdot \frac{1}{2} (1 - \cos(2 \theta)) = 2(1 - \cos(2 \theta)).
		\end{align*}
	The period of both $\sin(2 \theta)$ and $\cos(2 \theta)$ is $\pi$.  
	So the entire graph of the equation $r = 4 \sin \theta$ is traversed over the region $0 \leq \theta < \pi$.
	\end{freeResponse}

\end{problem}

\begin{instructorNotes}
Students sometimes have difficulty with the ``Cartesian to Polar'' graphing method given in the textbook.  
This will be important in the next section when they need to visualize curves in order to integrate in polar coordinates.
\end{instructorNotes}







%problem 8
\begin{problem}
Graph $r = 2 + 4 \cos \theta$ using the ``Cartesian-to-Polar'' method.
	\begin{freeResponse}
	First, we graph $r = 2 + 4 \cos \theta$ as if $r$ and $\theta$ were Cartesian coordinates.
	
	\begin{image}
	\includegraphics[trim= 170 310 170 290, scale=0.8]{Figure11-2-10.pdf}
	\end{image}
	
	We then use this to draw the following graph in the $xy$-plane
	
	\begin{image}
	\includegraphics[trim= 170 270 170 270, scale=0.7]{Figure11-2-11.pdf}
	\end{image}
	
	If it helps, here are the parametric equations for this graph
		\begin{align*}
		&x = r \cos \theta = (2 + 4 \cos \theta) \cos \theta = 2 \cos \theta + 4 \cos^2 \theta = 2(\cos \theta + 1 + \cos(2 \theta)) \\
		&y = 4 \sin \theta = (2 + 4 \cos \theta) \sin \theta = 2 \sin \theta + 2 \sin (2 \theta).
		\end{align*}
	
	\end{freeResponse}

\end{problem}

\begin{instructorNotes}
See problem 7.
\end{instructorNotes}








\end{document} 




