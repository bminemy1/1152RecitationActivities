\documentclass[handout]{ximera}
%handout:  for handout version with no solutions or instructor notes
%handout,instructornotes:  for instructor version with just problems and notes, no solutions
%noinstructornotes:  shows only problem and solutions

%% handout
%% space
%% newpage
%% numbers
%% nooutcomes

%I added the commands here so that I would't have to keep looking them up
%\newcommand{\RR}{\mathbb R}
%\renewcommand{\d}{\,d}
%\newcommand{\dd}[2][]{\frac{d #1}{d #2}}
%\renewcommand{\l}{\ell}
%\newcommand{\ddx}{\frac{d}{dx}}
%\everymath{\displaystyle}
%\newcommand{\dfn}{\textbf}
%\newcommand{\eval}[1]{\bigg[ #1 \bigg]}

%\begin{image}
%\includegraphics[trim= 170 420 250 180]{Figure1.pdf}
%\end{image}

%add a ``.'' below when used in a specific directory.
\input{../preamble.tex} %% we can turn off input when making a master document

\usepackage{fullpage}

\title{Recitation \# 11: Improper Integrals and Differential Equations}  

\begin{document}
\begin{abstract}		\end{abstract}
\maketitle




\section{Warm up:}
%\begin{problem}
True or False:  It is possible for a region to be infinitely long but have a finite area.
	\begin{freeResponse}
	True.  Consider the region below the curve $y=\frac{1}{x^2}$, $x \geq 1$.
	\end{freeResponse}
	
\begin{instructorNotes}

\end{instructorNotes}
%\end{problem}





\section{Group work:}


% Problem 1
\begin{problem}
Review of limits:
	\begin{enumerate}
	
	\item  $\lim_{x \to -\infty} \left( 3x^{-6} + e^{5x} + \frac{\sin x}{x^2 + 3} \right)$
	\begin{freeResponse}
	Recall that the limit of a sum is the sum of the limits, provided that those limits exist.
		\begin{itemize}
		\item  $\lim_{x \to -\infty} 3x^{-6} = \lim_{x \to -\infty} \frac{3}{x^6} = 0$.
		\item  $\lim_{x \to -\infty} e^{5x} = 0$.
		\item  $\lim_{x \to -\infty} \frac{\sin x}{x^2 + 3} = 0$  
		
		\vspace{8pt}
		
		{\color{red} \text{To rigorously prove this, you need to use the squeeze theorem.}}
		\end{itemize}
	Thus, 
		\[
		\lim_{x \to -\infty} \left( 3x^{-6} + e^{5x} + \frac{\sin x}{x^2 + 3} \right) = 0
		\]
	\end{freeResponse}
	
	
	
	\item  $\lim_{x \to \infty} \frac{x}{\sqrt{9x^2+4}}$
	\begin{freeResponse}
		\begin{align*}
		\lim_{x \to \infty} \frac{x}{\sqrt{9x^2+4}}
		&= \lim_{x \to \infty} \frac{x}{\sqrt{x^2} \cdot {\sqrt{9 + \frac{4}{x^2}}}}  \\
		&= \lim_{x \to \infty} \frac{x}{|x| \cdot \sqrt{9 + \frac{4}{x^2}}}  \\
		&= \lim_{x \to \infty} \frac{x}{x \cdot \sqrt{9 + \frac{4}{x^2}}}  \\
		&= \lim_{x \to \infty} \frac{1}{\sqrt{9 + \frac{4}{x^2}}}  \\
		&= \frac{1}{\sqrt{9 + 0}} = \frac{1}{3}.
		\end{align*}
	\end{freeResponse}
	
	
	
	\item  $\lim_{x \to -\infty} \arctan x$
	\begin{freeResponse}
	$\lim_{x \to -\infty} \arctan x = - \frac{\pi}{2}. $
	\end{freeResponse}
	
	\end{enumerate}
		
\end{problem}

\begin{instructorNotes}
Review of limits.
\end{instructorNotes}








%problem 2
\begin{problem}
Determine if the given integral converges or diverges.  
If it converges, find the value.
	
	 $\int_{-1}^{\infty} \frac{3}{2x+1} \d x$
	\begin{freeResponse}
	The function $\frac{3}{2x+1}$ has a vertical asymptote at $x=- \frac{1}{2}$.  
	So we rewrite the original integral as
		\[
		\int_{-1}^{\infty} \frac{3}{2x+1} \d x = \lim_{a \to -\frac{1}{2}^-} \int_{-1}^a \frac{3}{2x+1} \d x  
		+ \lim_{b \to -\frac{1}{2}^+} \int_b^0 \frac{3}{2x+1} \d x + \lim_{c \to \infty} \int_0^c \frac{3}{2x+1} \d x.
		\]
	The latter integral does not exist.  
	To see this, just note that
		\begin{align*}
		 \lim_{c \to \infty} \int_0^c \frac{3}{2x+1} \d x 
		 &= \lim_{c \to \infty} \eval{\frac{3}{2} \ln|2x+1|}_0^c  \\
		 &= \lim_{c \to \infty} \frac{3}{2} \ln|2c+1| = \infty.
		\end{align*}
	Therefore, 
		\[
		\int_{-1}^{\infty} \frac{3}{2x+1} \d x \; \text{ diverges.}
		\]
	\end{freeResponse}

	
%	\item  $\int_{-\infty}^{\infty} x e^{-x} \d x$
%	\begin{freeResponse}
%	\[
%	\int_{-\infty}^{\infty} x e^{-x} \d x = \lim_{a \to -\infty} \int_a^0 xe^{-x} \d x + \lim_{b \to \infty} \int_0^b xe^{-x} \d x.
%	\]
%	The first of these integrals does not exist.  
%	To see this, we apply integration by parts
%		\begin{align*}
%		\lim_{a \to -\infty} \int_a^0 xe^{-x} \d x
%		&= \lim_{a \to -\infty} \left( \eval{-xe^{-x}}_a^0 + \int_a^0 e^{-x} \d x \right)  \\
%		&= \lim_{a \to -\infty} \left( \left[ 0 + ae^{-a} \right] + \eval{-e^{-x}}_a^0 \right)  \\
%		&= \lim_{a \to -\infty} \left( ae^{-a} + e^{-a} - 1 \right)  \\
%		&= \lim_{a \to -\infty} \left( e^{-a}(a+1) - 1 \right)  \\
%		&= - \infty.
%		\end{align*}
%	Thus
%		\[
%		\int_{-\infty}^{\infty} x e^{-x} \d x \; \text{ diverges.}
%		\]
%	\end{freeResponse}
%	
%	
%	
%	\item  $\int_6^{\infty} \frac{2-4x}{2x^2 - 13x + 20} \d x$
%	\begin{freeResponse}
%	First notice that we can factor the denominator as
%		\[
%		2x^2 - 13x + 20 = (2x-5)(x-4).
%		\]
%	So we find a partial fraction decomposition of the integrand
%		\begin{align*}
%		&\frac{2-4x}{(2x-5)(x-4)} = \frac{A}{2x-5} + \frac{B}{x-4}  \\
%		\Longrightarrow \qquad &2-4x = A(x-4) + B(2x-5).
%		\end{align*}
%	We can solve for $A$ and $B$ via the following substitutions:
%		\begin{itemize}
%		\item {\color{red}$\left( \text{Let } x=\frac{5}{2} \right)$} \quad $2 - 4 \left(\frac{5}{2} \right) = A \left(\frac{5}{2} - 4 \right)$  \\
%			\[
%			-8 = - \frac{3}{2} A \qquad \Longrightarrow \qquad A = \frac{16}{3}.
%			\]
%		\item {\color{red} (Let $x=4$)}  \quad $2-16 = 3B \qquad \Longrightarrow \qquad B = - \frac{14}{3}$.
%		\end{itemize}
%	Thus, 
%		\begin{align*}
%		\int_6^{\infty} \frac{2-4x}{2x^2 - 13x + 20} \d x
%		&= \lim_{a \to \infty} \int_6^a \left[ \frac{\frac{16}{3}}{2x-5} - \frac{\frac{14}{3}}{x-4} \right] \d x  \\
%		&= \lim_{a \to \infty} \eval{ \frac{8}{3} \ln|2x-5| - \frac{14}{3} \ln|x-4| }_6^a  \\
%		&= \lim_{a \to \infty} \left[ \frac{8}{3} \left( \ln|2a-5| - \ln(7) \right) - \frac{14}{3} \left( \ln|a-4| - \ln(2) \right) \right]  \\
%		&= \lim_{a \to \infty} \left[ \frac{1}{3} \left( \ln(2a-5)^8 - \ln(a-4)^{14} \right) - \frac{8}{3} \ln (7) + \frac{14}{3} \ln(2) \right]  \\
%		&= \lim_{a \to \infty} \left[ \frac{1}{3} \ln \left( \frac{(2a-5)^8}{(a-4)^{14}} \right) - \frac{8}{3} \ln (7) + \frac{14}{3} \ln(2) \right]  \\
%		&= - \infty.
%		\end{align*}
%	Thus,
%		\[
%		\int_6^{\infty} \frac{2-4x}{2x^2 - 13x + 20} \d x \; \text{ diverges.}
%		\]
%	\end{freeResponse}
	
	
\end{problem}

\begin{instructorNotes}
Make sure that students write in the limit notation throughout their work, with the limit taken after the definite integral has been evaluated.
	\begin{enumerate}
	\item has a vertical asymptote as well as a limit going to infinity.
	\item is by parts, and L'Hospital's Rule will be useful for the limit.
	\item will need partial fractions.  
	The coefficients of the decomposition are rigged to be opposites of each other, so that one can use properties of logarithms to aid in taking the limit.
	\end{enumerate}
\end{instructorNotes}







%%problem 
%\begin{problem}
%Find the volume of the solid whose base is the region where $x \geq 1$, $y \geq 0$, and below the curve $y=\frac{1}{x^4}$, and whose cross sections perpendicular to the $x$-axis are squares.
%	\begin{freeResponse}
%	The area of each cross-section is $\left( \frac{1}{x^4} \right)^2 = \frac{1}{x^8}$.  
%	Thus,
%		\begin{align*}
%		{\color{red}\text{Volume }} &= \int_1^{\infty} \frac{1}{x^8} \d x  \\
%		&= \lim_{a \to \infty} \eval{- \frac{1}{7 x^7}}_1^a  \\
%		&= \lim_{a \to \infty} \left( - \frac{1}{7a^7} + \frac{1}{7} \right)  \\
%		&= 0 + \frac{1}{7} = \frac{1}{7}.
%		\end{align*}
%	\end{freeResponse}
%
%\end{problem}
%
%\begin{instructorNotes}
%
%\end{instructorNotes}




%
%%problem 2
%\begin{problem}
%Verify that, if $y(0)=0$, that both $f(x)=1-(x^2+1)^2$ \dfn{and} $g(x) = 1 - (x^2-1)^2$ are solutions to the differential equation $\dd[y]{x} = 4x\sqrt{1-y}$.
%	\begin{freeResponse}
%		\begin{align*}
%		f'(x) &= -2(x^2+1) \cdot 2x  \\
%		&= -4x \sqrt{(x^2+1)^2}  \\
%		&= -4x \sqrt{1-1+(x^2+1)^2}  \\
%		&= -4x \sqrt{1 - \left[1-(x^2+1)^2 \right]}  \\
%		&= -4x \sqrt{1-f(x)}.
%		\end{align*}
%		
%		\vskip 5pt
%		
%		\begin{align*}
%		g'(x) &= -2(x^2-1) \cdot 2x  \\
%		&= -4x \sqrt{(x^2-1)^2}  \\
%		&= -4x \sqrt{1-1+(x^2-1)^2}  \\
%		&= -4x \sqrt{1 - \left[1-(x^2-1)^2 \right]}  \\
%		&= -4x \sqrt{1-g(x)}.
%		\end{align*}
%	\end{freeResponse}
%		
%\end{problem}
%
%\begin{instructorNotes}
%Note that students will need to plug into both $x$ and $y$ in the differential equation.  
%Also, make sure that students realize that an initial value problem can possibly have many solutions.
%\end{instructorNotes}




%problem 3
\begin{problem}
Which of the following is a solution to the differential equation $y'' + 9y = 0$?
	\begin{enumerate}
	\item  $y=e^{3t}+e^{-3t}$
	\item  $y=C(t^2 + t)$
	\item  $y=\sin(3t) + 6$
	\item  $y=5 \cos(3t) - 7 \sin(3t)$
	\item  $y=A \cos(3t) + B \sin(3t)$ \text{ (where A and B are real numbers.)}
	\end{enumerate}
	
	\begin{freeResponse}
	\begin{enumerate}
	
	\item  \[ y = e^{3t}+e^{-3t} 	\qquad	y'=3e^{3t}-3e^{-3t} 	\qquad	y''=9e^{3t}+9e^{-3t} \]  
	So,
		\begin{align*}
		y''+9y &= (9e^{3t}+9e^{-3t}) + 9\cdot (e^{3t}+e^{-3t})  \\
		&= 18e^{3t} + 18e^{-3t} \neq 0.
		\end{align*}
	Therefore, this is \dfn{not} a solution to $y''+9y=0$.
	
	
	
	\item  \[ y = C(t^2+t) 	\qquad	y'=C(2t+1) 	\qquad	y''=2C \] 
	So,
		\begin{align*}
		y''+9y &= 2C + 9C(t^2+t)  \\
		&= 9Ct^2 + 9Ct + 2C \neq 0 \text{ if } C \neq 0.
		\end{align*}
	Therefore, this is \dfn{not} a solution to $y''+9y=0$ unless $C=0$, in which case we get the {\it trivial} solution.
	
	
	
	\item  \[ y = \sin(3t)+6 	\qquad	y'=3\cos(3t) 	\qquad	y''=-9\sin(3t) \]  
	So,
		\begin{align*}
		y''+9y 
		&= -9\sin(3t) +9(\sin(3t) + 6)  \\
		&= 54 \neq 0.
		\end{align*}
	Therefore, this is \dfn{not} a solution to $y''+9y=0$.
	
	
	
	\item  \[ y = 5\cos(3t)-7\sin(3t) 	\qquad	y'=-15\sin(3t)-21\cos(3t) 	\qquad	y''=-45\cos(3t)+63\sin(3t) \]  
	So,
		\begin{align*}
		y''+9y 
		&= -45\cos(3t)+63\sin(3t) +9(5\cos(3t)-7\sin(3t))  \\
		&= -45\cos(3t)+63\sin(3t) + 45\cos(3t) - 63\sin(3t) = 0.
		\end{align*}
	Therefore, this \dfn{is} a solution to $y''+9y=0$.
	
	
	
	\item  \[ y = A\cos(3t)+B\sin(3t) 	\qquad	y'=-3A\sin(3t)+3B\cos(3t) 	\qquad	y''=-9A\cos(3t)-9B\sin(3t) \]  
	So,
		\begin{align*}
		y''+9y 
		&= -9A\cos(3t)-9B\sin(3t) +9(A\cos(3t)+B\sin(3t))  \\
		&= 0.
		\end{align*}
	Therefore, this \dfn{is} a solution to $y''+9y=0$.
	
	\end{enumerate}
	\end{freeResponse}
	
%problem 4
\begin{problem}
Explain why the functions with the given graphs cannot be solutions of the differential equation $y' = e^x (y-1)^2$.
	\begin{image}
	\includegraphics[trim= 170 200 190 180, scale=0.8, angle=-88.69]{Figure8-1-1.pdf}	
	\end{image}

	\begin{freeResponse}
	Since $y' = e^x (y-1)^2$, the derivative of $y$ is always nonnegative.  
	Thus, the first graph cannot satisfy this differential equation since it has a tangent line with a negative slope.
	
	\begin{image}
	\includegraphics[trim= 170 300 190 310, scale=1]{Figure8-1-2.pdf}	
	\end{image}
	
	The second graph cannot satisfy the differential equation since the slope of the tangent line at $x=1$ is positive
	
	\begin{image}
	\includegraphics[trim= 170 330 190 330, scale=1]{Figure8-1-3.pdf}	
	\end{image}
	
	but $\eval{\dd[y]{x}}_{x=0} = e^0(1-1) = 0.$
	\end{freeResponse}

\end{problem}

\begin{instructorNotes}
This activity anticipates the idea of direction fields to be covered in Section 8.2.  
Students should eventually realize that differrential equations are statements about the slope of the function at each point $(x,y)$.  
Issues such as where the slope equals $0$ or where the slope is positive, negative, increasing, or decreasing should arise in the solution/discussion.  
The first graph has a negative slope at some places ($e^x$ and $(y-1)^2$ are always non-negative), and the second graph does not have a slope of $0$ at the point $(1,1)$.  
\end{instructorNotes}

\begin{instructorNotes}
This is a good exercise in evaluating differential equations with functions.  
Suggest that each member in the group try a different possible solution.
\end{instructorNotes}

\end{problem}














	
	
	
	
	
	
	
	
	

	










								
				
				
	







	
	
	
	
	
	
	
	
	

	










								
				
				
	














\end{document} 


















