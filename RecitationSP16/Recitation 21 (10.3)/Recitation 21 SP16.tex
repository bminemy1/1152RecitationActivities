\documentclass[noinstructornotes]{ximera}
%handout:  for handout version with no solutions or instructor notes
%handout,instructornotes:  for instructor version with just problems and notes, no solutions
%noinstructornotes:  shows only problem and solutions

%% handout
%% space
%% newpage
%% numbers
%% nooutcomes

%I added the commands here so that I would't have to keep looking them up
%\newcommand{\RR}{\mathbb R}
%\renewcommand{\d}{\,d}
%\newcommand{\dd}[2][]{\frac{d #1}{d #2}}
%\renewcommand{\l}{\ell}
%\newcommand{\ddx}{\frac{d}{dx}}
%\everymath{\displaystyle}
%\newcommand{\dfn}{\textbf}
%\newcommand{\eval}[1]{\bigg[ #1 \bigg]}

%\begin{image}
%\includegraphics[trim= 170 420 250 180]{Figure1.pdf}
%\end{image}

%add a ``.'' below when used in a specific directory.
\input{../preamble.tex} %% we can turn off input when making a master document
\usepackage{fullpage}


\title{Recitation \#21: Taylor series}  

\begin{document}
\begin{abstract}		\end{abstract}
\maketitle



\section{Warm up:}
Find the Taylor series for:  
	\begin{enumerate}
	\item $27x^2 - 3x + 17$ centered at $a=0$.  
	\item  $27x^2 - 3x + 17$ centered at $a=1$.  
	\end{enumerate}
	
	\begin{freeResponse}
	\begin{enumerate}
	\item
	$27x^2 - 3x + 17$ is already a Taylor Series centered at 0 because it is already in the form $\sum_{k=0}^\infty c_k x^k$ with $c_0=17, c_1=-3, c_2=27$, and the rest of the $c_k=0$.
	
	\item  
	Let $f(x) = 27x^2-3x+17$.  Then
		\begin{align*}
		&f(1) = 27 - 3 + 17 = 41  \\
		&f'(x) = 54x - 3 	\qquad \Longrightarrow 	\qquad	f'(1) = 54-3=51  \\
		&f''(x) = 54 		\qquad	\Longrightarrow	\qquad f''(1) = 54  \\
		&f^{(3)}(x) = 0 	\qquad	\Longrightarrow	\qquad f^{(3)}(1) = 0  \\
		&\qquad \vdots 	\qquad	\qquad	\qquad \qquad \qquad	\vdots  \\
		&f^{(n)}(x) = 0	\qquad	\Longrightarrow	\qquad	f^{(n)}(1) = 0.
		\end{align*}
	So
		\[
		f(x) = \sum_{k=0}^\infty \frac{f^{(k)}(a)}{k!}(x-a)^k = \boxed{41 + 51(x-1) + \frac{54}{2!}(x-1)^2}
		\]
	Lastly, note that if you multiply this out then you will get back the original polynomial.
	
	
	

	
	\end{enumerate}
	\end{freeResponse}
	
\begin{instructorNotes}
Here, they need to compute the Taylor series by computing derivatives and recognizing patterns.  

Part (a) is an opportunity to show the students that a polynomial is already a Maclaurin series.  
Use derivatives to figure out the Taylor series about $a=1$, and then show them that the answer simplifies back to the original problem.  

Part (b) is a continuation from the previous recitation where they already found the approximating polynomial.  
The students may have a problem finding the pattern in the derivatives (especially with the alternating part).
\end{instructorNotes}







\section{Group work:}




%problem 1
\begin{problem}
Find a Maclaurin series (and interval of convergence) for
\[
f(x) = x^3 \sin(x^5)
\]
	
	\begin{freeResponse}
	
  We already know that 
		\[
		\sin x = \sum_{k=0}^\infty \frac{(-1)^k x^{2k+1}}{(2k+1)!}
		\]
	with interval of convergence $( - \infty, \infty)$.  
	So we use this to compute
		\begin{align*}
		x^3 \sin(x^5) &= x^3 \sum_{k=0}^\infty \frac{(-1)^k (x^5)^{2k+1}}{(2k+1)!}  \\
		&= x^3 \sum_{k=0}^\infty \frac{(-1)^k x^{10k+5}}{(2k+1)!}  \\
		&= \boxed{\sum_{k=0}^\infty \frac{(-1)^k x^{10k+8}}{(2k+1)!}}
		\end{align*}
	with interval of convergence $( - \infty, \infty)$.  
	
	

	\end{freeResponse}
	
\end{problem}

\begin{instructorNotes}
Students should use the known Maclaurin series in various ways.  
You might want to give the hint in part (d) that $\ddx \sin^{-1}(x) = \frac{1}{\sqrt{1-x^2}}$.  
\end{instructorNotes}



%problem 2
\begin{problem}
Find the first four non-zero terms of the Maclaurin Series for 
\[
 xe^{x^2}+\cos(x^3)
\]

\begin {freeResponse}
We might be tempted to solve this problem using the definition of a Taylor Series since we only need the first four terms, but that is a bad idea because the derivatives will get very messy.  Instead, we will use the known Maclaurin Series for $e^x$ and $\cos(x)$.

We already know that 
		\[
		e^x = \sum_{k=0}^\infty \frac{(x)^k}{k!}
		\]
	with interval of convergence $( - \infty, \infty)$.  
	So we use this to compute
		\begin{align*}
		x e^{x^2} &= x \sum_{k=0}^\infty \frac{(x^2)^k}{k!}  \\
		&= x \sum_{k=0}^\infty \frac{x^{2k}}{k!}  \\
		&= \sum_{k=0}^\infty \frac{x^{2k+1}}{k!}
		\end{align*}
	with interval of convergence $( - \infty, \infty)$.  
	
Similarly, we already know that 
		\[
		\cos(x) = \sum_{k=0}^\infty \frac{(-1)^k x^{2k}}{(2k)!}
		\]
	with interval of convergence $( - \infty, \infty)$.  
	So we use this to compute
		\begin{align*}
		\cos(x^3) &=  \sum_{k=0}^\infty \frac{(-1)^k (x^3)^{2k}}{(2k)!}  \\
		&= \sum_{k=0}^\infty \frac{(-1)^k x^{6k}}{(2k)!}
		\end{align*}
	with interval of convergence $( - \infty, \infty)$.  
	
	Therefore, 
	\begin{align*}
	 xe^{x^2}+\cos(x^3) &= \sum_{k=0}^\infty \frac{x^{2k+1}}{k!} + \sum_{k=0}^\infty \frac{(-1)^k x^{6k}}{(2k)!}
	 &= \sum_{k=0}^\infty \left(\frac{x^{2k+1}}{k!} + \frac{(-1)^k x^{6k}}{(2k)!}\right)
	 \end{align*}

Now, we just need to plug in numbers for k, starting with k=0, until we have four non-zero terms. Plugging in 0 through 3 gives:\\
\begin{align*}
&\left(\frac{x^{2(0)+1}}{(0)!} + \frac{(-1)^{(0)} x^{6(0)}}{(2(0))!}\right)\\ 
+ &\left(\frac{x^{2(1)+1}}{(1)!} + \frac{(-1)^{(1)} x^{6(1)}}{(2(1))!}\right)\\
+ &\left(\frac{x^{2(2)+1}}{(2)!} + \frac{(-1)^{(2)} x^{6(2)}}{(2(2))!}\right) \\
+ &\left(\frac{x^{2(3)+1}}{(3)!} + \frac{(-1)^{(3)} x^{6(3)}}{(2(3))!}\right)
\end{align*}
Now, notice that you actually have more than four terms here because there are more than four powers of $x$.  We will have to simplify thisa expression to get the first four terms.

$x + 1 + x^{3} + \frac{-x^{6}}{2} + \frac{x^{5}}{2} + \frac{x^{12}}{4}
+ \frac{x^{7}}{6} + \frac{- x^{18}}{12}$ \\

so the first four terms will be

$\boxed{1 + x + x^{3} + \frac{x^{5}}{2}}$

\end{freeResponse}


\end{problem}



%problem 3
\begin{problem}
Find a function (closed expression) for the following series and the interval on which the function and the series are equal.
	\[
	x + x^4 + \frac{1}{2} x^7 + \frac{1}{6} x^{10} + \frac{1}{24} x^{13} + \hdots
	\]
	\begin{freeResponse}
		\begin{align*}
		x + x^4 + \frac{1}{2} x^7 + \frac{1}{6} x^{10} + \frac{1}{24} x^{13} + \hdots
		&= x + x^4 + \frac{1}{2!} x^7 + \frac{1}{3!} x^{10} + \frac{1}{4!} x^{13} + \hdots  \\
		&= \sum_{k=0}^\infty \frac{1}{k!}x^{3k + 1}  \\
		&= x \sum_{k=0}^\infty \frac{x^{3k}}{k!}  \\
		&= x \sum_{k=0}^\infty \frac{(x^3)^k}{k!}  \\
		&= \boxed{xe^{x^3}}
		\end{align*}
	which has interval of convergence $(- \infty, \infty)$.  
	\end{freeResponse}
		
\end{problem}

\begin{instructorNotes}
The students need to rewrite $f(x)$ in summation notation (factoring out an $x$) and seeing the series for $xe^{x^3}$.  
\end{instructorNotes}







%problem 4
\begin{problem}
Compute the sum of the following series ({\it Hint:  You should use Taylor series.})
	\begin{enumerate}
	\item  $1 - \ln 2 + \frac{(\ln 2)^2}{2!} - \frac{(\ln 2)^3}{3!} + \hdots$
	\item  $3 + \frac{9}{2!} + \frac{27}{3!} + \frac{81}{4!} + \hdots$
	\end{enumerate}
	
	\begin{freeResponse}
	\begin{enumerate}
	\item  
		\begin{align*}
		1 - \ln 2 + \frac{(\ln 2)^2}{2!} - \frac{(\ln 2)^3}{3!} + \hdots
		&= \sum_{k=0}^\infty \frac{(- \ln 2)^k}{k!}  \\
		&= e^{- \ln 2}  = e^{\ln 2^{-1}}  \\
		&= 2^{-1} = \boxed{\frac{1}{2}}.
		\end{align*}
	
	
	
	\item  
		\begin{align*}
		3 + \frac{9}{2!} + \frac{27}{3!} + \frac{81}{4!} + \hdots
		&= \sum_{k=1}^\infty \frac{3^k}{k!}  \\
		&= \sum_{k=0}^\infty \frac{3^k}{k!} - \frac{3^0}{0!}  \\
		&= \boxed{e^3 - 1}.
		\end{align*}
	
	\end{enumerate}
	\end{freeResponse}

\end{problem}

\begin{instructorNotes}
The goal here is for students to realize that Taylor series gives them a tool for finding the exact sum of a series.
\end{instructorNotes}


%problem
\begin{problem}
 Find the Taylor Series for $\sin(2x)$ about $a = \frac{\pi}{8}$.  \\
		Hint: Recall from a previous recitation that \\
		$p_3(x) = \frac{\sqrt{2}}{2} + \sqrt{2} \left( x - \frac{\pi}{8} \right) - \sqrt{2} \left( x - \frac{\pi}{8} \right)^2 -\frac{2 \sqrt{2}}{3} \left( x - \frac{\pi}{8} \right)^3 $
\begin{freeResponse}
	Let $f(x) = \sin(2x)$.  Then
		\begin{align*}
		&f \left( \frac{\pi}{8} \right) = \sin \left( \frac{2\pi}{8} \right) = \frac{\sqrt{2}}{2}  \\
		&f'(x) = 2 \cos(2x) 	\qquad \Longrightarrow 	\qquad	f'\left( \frac{\pi}{8} \right) = 2 \cdot \frac{\sqrt{2}}{2} = \sqrt{2}  \\
		&f''(x) = -4 \sin(2x) 		\qquad	\Longrightarrow	\qquad f''\left( \frac{\pi}{8} \right) = -4 \cdot \frac{\sqrt{2}}{2} = -2 \sqrt{2}  \\
		&f^{(3)}(x) = -8 \cos(2x)	\qquad	\Longrightarrow	\qquad f^{(3)}\left( \frac{\pi}{8} \right) = -8 \cdot \frac{\sqrt{2}}{2} = -4 \sqrt{2}  \\
		&f^{(4)}(x) = 16 \cos(2x) \qquad 	\Longrightarrow	\qquad f^{(4)}\left( \frac{\pi}{8} \right) =16 \cdot \frac{\sqrt{2}}{2}  = 8 \sqrt{2}.
		\end{align*}
	Continuing this pattern, we see that
		\[
		f^{(k)} \left( \frac{\pi}{8} \right) = (-1)^{\left\lceil \frac{k}{2} \right\rceil } 2^{k-1} \sqrt{2}
		\]
	where $\left\lceil \frac{k}{2} \right\rceil$ denotes the smallest integer greater than $\frac{k}{2}$.  
	So, for example, $\left\lceil \frac{1}{2} \right\rceil = 1$, $\left\lceil \frac{2}{2} \right\rceil = 1$, $\left\lceil \frac{3}{2} \right\rceil = 2$, and so on.  
	
	So from here we have that the Taylor series for $f(x)$ is
		\[
		\boxed{\sum_{k=0}^\infty \frac{(-1)^{\left\lceil \frac{k}{2} \right\rceil } 2^{k-1} \sqrt{2}}{k!} \left( x - \frac{\pi}{8} \right)^k}
		\]
\end{freeResponse}

\end{problem}












	
	
	
	
	
	
	
	
	

	










								
				
				
	














\end{document} 


















