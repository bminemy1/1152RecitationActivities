\documentclass[noinstructornotes]{ximera}
%handout:  for handout version with no solutions or instructor notes
%handout,instructornotes:  for instructor version with just problems and notes, no solutions
%noinstructornotes:  shows only problem and solutions

%% handout
%% space
%% newpage
%% numbers
%% nooutcomes

%I added the commands here so that I would't have to keep looking them up
%\newcommand{\RR}{\mathbb R}
%\renewcommand{\d}{\,d}
%\newcommand{\dd}[2][]{\frac{d #1}{d #2}}
%\renewcommand{\l}{\ell}
%\newcommand{\ddx}{\frac{d}{dx}}
%\everymath{\displaystyle}
%\newcommand{\dfn}{\textbf}
%\newcommand{\eval}[1]{\bigg[ #1 \bigg]}

%\begin{image}
%\includegraphics[trim= 170 420 250 180]{Figure1.pdf}
%\end{image}

%add a ``.'' below when used in a specific directory.
\input{../preamble.tex} %% we can turn off input when making a master document
<<<<<<< HEAD
=======
\usepackage{fullpage}
>>>>>>> origin/master

\title{Recitation \#14: Sequences}  

\begin{document}
\begin{abstract}		\end{abstract}
\maketitle



\section{Warm up:}
Find the limit of the following sequences as $n$ tends to $\infty$. 
\begin{enumerate}

\item $a_n = \frac{n^{1000}}{2^n}$

	
\item $b_n = \cos (n \pi)$

\item $c_n = \cos (n! \pi)$

\end{enumerate}
	\begin{freeResponse}
\begin{enumerate}	
\item Note that $\lim_{x \to \infty} \frac{x^a}{b^x} = 0$ for any constants $a$ and $b>1$. So $\lim_{n \to \infty} a_n =0$.
\item If $n$ is even, $b_n = \cos(n \pi) = 1$, but if $n$ is odd, then $b_n = \cos(n \pi) = -1$. So $\lim_{n \to \infty} b_n$ does not exist.
\item If $n$ is at least 2, then $n!$ is even. So $c_n = 1$ if $n$ is at least 2. $\lim_{n \to \infty} c_n = 1$. 
\end{enumerate}
	\end{freeResponse}
\begin{instructorNotes}

\end{instructorNotes}







\section{Group work:}



<<<<<<< HEAD
%problem 1
\begin{problem}
For each of the following sequences, find the limit as the number of terms approaches infinity.
	\begin{enumerate}
=======



%problem 1
\begin{problem}
Find the limit of the given sequence.  
Also, determine if it is a geometric sequence.
	\begin{enumerate}
	\begin{multicols}{3}
	
	
	\item  $a_n = \frac{n^2}{2^n}$
	%\begin{freeResponse}
	
	%\end{freeResponse}
	
	
	
	\item  $a_n = \frac{1}{3^n}$
	%\begin{freeResponse}
	
	%\end{freeResponse}
	
	
	
	\item  $a_n = \left( \frac{1}{n} \right)^4$
	%\begin{freeResponse}
	
	%\end{freeResponse}
	
	
	
	\item  $a_n = \frac{e^n + (-3)^n}{5^n}$
	%\begin{freeResponse}
	
	%\end{freeResponse}
	
	
	
	\item  $a_n = 3^{\frac{1}{n}}$
	%\begin{freeResponse}
	
	%\end{freeResponse}
	
	
	\end{multicols}
	\end{enumerate}
	
	\begin{freeResponse}
	\begin{enumerate}
	\item 	$\lim_{n \to \infty} \frac{n^2}{2^n} = 0 	\quad	{\color{red}\text{growth rate}}$
	
	\item  $\lim_{n \to \infty} \frac{1}{3^n} = \lim_{n \to \infty} \left( \frac{1}{3} \right)^n = 0.$
	This is a geometric sequence with $a = 1$ and $r = \frac{1}{3}$.  
	
	\item  $\lim_{n \to \infty} \left( \frac{1}{n} \right)^4 = 0$.  
	
	\item  $\lim_{n \to \infty} \frac{e^n + (-3)^n}{5^n} = \lim_{n \to \infty} \left[ \left( \frac{e}{5} \right)^n + \left( \frac{-3}{5} \right)^n \right] = 0.$  
	
	This is the sum of two geometric sequences.  
	For both, the initial term is $a = 1$.  
	For the first sequence the ratio is $r_1 = \frac{e}{5}$, and for the second the ratio is $r_2 = \frac{-3}{5}$.
	
	\item  $\lim_{n \to \infty} 3^\frac{1}{n} = 3^0 = 1$.  
	
	\end{enumerate}
	\end{freeResponse}

\end{problem}

\begin{instructorNotes}
These limits should be relatively easy to analyze.  
The students need to identify the ``$r$'' if it is a geometric sequence (and note that the exponent $n$ is the variable).  
On (d) and (f), they should argue that they are looking at the sum of two geometric sequences.  
Maybe give one per group with about $8$ minutes for discussion.  
\end{instructorNotes}







%problem 2
\begin{problem}
Show that 
$$\lim_{n \to \infty} \left( \sqrt{n+1} - \sqrt{n} \right)$$ 
exists by proving that $a_n = \sqrt{n+1} - \sqrt{n}$ is a bounded monotonic sequence.  A hint is to show that $f(x) = \sqrt{x+1} - \sqrt{x}$ is a decreasing function by showing that $f'(x) < 0$.  
	\begin{freeResponse}
	Let $f(x) = \sqrt{x+1} - \sqrt{x}$.  Then
		\begin{align*}
		f'(x) 
		&= \frac{1}{2 \sqrt{x+1}} - \frac{1}{2 \sqrt{x}}  \\
		&= \frac{\sqrt{x} - \sqrt{x+1}}{2 \sqrt{x}\sqrt{x+1}}  \\
		&< 0
		\end{align*}
	since the denominator is clearly positive, and $\sqrt{x} < \sqrt{x+1}$.
	Therefore $f$ is decreasing, and so the original sequence is decreasing.  
	Also notice that since 
	$$\sqrt{x} < \sqrt{x+1}$$
	we have that 
	$$0 < \sqrt{x+1} - \sqrt{x} = f(x).$$
	Thus the original sequence is bounded below by $0$.  \

	Therefore, since the sequence $\left\{ \sqrt{n+1} - \sqrt{n} \right\}$ is bounded and monotone decreasing, the limit
		\[
		\lim_{n \to \infty} \sqrt{n+1} - \sqrt{n}
		\]
	exists.
	\end{freeResponse}
		
\end{problem}

\begin{instructorNotes}
Perhaps do as a whole class discussion.  
Emphasize careful writing of reasoning.
\end{instructorNotes}




%problem 3
\begin{problem}
For each of the following sequences, find the limit as the number of terms approaches infinity.
	\begin{enumerate}
	%\begin{multicols}{2}
>>>>>>> origin/master
	
	\item  $a_n = \left( \frac{n+1}{2n} \right) \left( \frac{n-2}{n} \right)^{\frac{n}{2}}$
	\begin{freeResponse}
	Let $f(x) =  \left( \frac{x+1}{2x} \right) \left( \frac{x-2}{x} \right)^{\frac{x}{2}}$.  
	Then
		\begin{align*}
		\lim_{x \to \infty} f(x) 
		&= \lim_{x \to \infty} e^{\ln f(x)}  \\
		&= e^{\lim_{x \to \infty} \ln f(x)}.
		\end{align*}
	So we need to compute the limit in the exponent.  To this end
		\begin{align*}
		\lim_{x \to \infty} \ln f(x) 
		&= \lim_{x \to \infty} \left[ \ln \left( \frac{x+1}{2x} \right) + \ln \left( \frac{x-2}{x} \right)^{\frac{x}{2}} \right]  \\
		&= \lim_{x \to \infty} \ln \left( \frac{x+1}{2x} \right) + \lim_{x \to \infty} \left[ \frac{x}{2} \ln \left( \frac{x-2}{x} \right) \right]  \quad {\color{red}\text{provided both limits exist}}  \\
		&= \ln \left( \frac{1}{2} \right) + \lim_{x \to \infty} \frac{\ln \left( 1 - \frac{2}{x} \right)}{\frac{2}{x}}  \quad {\color{red}\text{indeterminant of the form }\frac{0}{0}}\\
		&= \ln \left( \frac{1}{2} \right) + \lim_{x \to \infty} \frac{\frac{2x^{-2}}{1-\frac{2}{x}}}{-2x^{-2}}  \quad {\color{red}\text{L'Hospital's Rule}}  \\
		&= \ln \left( \frac{1}{2} \right) + \lim_{x \to \infty} \frac{-1}{1-\frac{2}{x}}  \\
		&= \ln \left( \frac{1}{2} \right) - 1.
		\end{align*}
	So 
		\[
		\lim_{x \to \infty} f(x) = e^{\ln \left( \frac{1}{2} \right) - 1} = \frac{1}{2} e^{-1}
		\]
	and therefore
		\[
		\lim_{n \to \infty} \left( \frac{n+1}{2n} \right) \left( \frac{n-2}{n} \right)^{\frac{n}{2}} = \frac{1}{2} e^{-1}.
		\]
	\end{freeResponse}
	
	
	
	\item  $a_n = \sqrt[n]{3^{2n+1}}$
	\begin{freeResponse}
		\begin{align*}
		\lim_{n \to \infty} \sqrt[n]{3^{2n+1}}
		&= \lim_{n \to \infty} \left( 3^{2n+1} \right)^{\frac{1}{n}}  \\
		&= \lim_{n \to \infty} 3^{2 + \frac{1}{n}}  \\
		&= \lim_{n \to \infty} 3^2 \cdot 3^{\frac{1}{n}}  \\
		&= 9 \lim_{n \to \infty} e^{\frac{1}{n}}  \\
		&= 9 \cdot 3^0  \\
		&= 9 \cdot 1 = 9.
		\end{align*}
	\end{freeResponse}
	
	
	
	\item  $a_n = \left( \sqrt{n^2+7} - n \right)$
	\begin{freeResponse}
		\begin{align*}
		\lim_{n \to \infty} \left( \sqrt{n^2+7} - n \right)
		&= \lim_{n \to \infty} \left[ \left( \sqrt{n^2 + 7} - n \right) \cdot \frac{\sqrt{n^2 + 7} + n}{\sqrt{n^2 + 7} + n} \right]  \\
		&= \lim_{n \to \infty} \frac{n^2 + 7 - n^2}{n \sqrt{1 + \frac{7}{n^2}} + n}  \\
		&= \lim_{n \to \infty} \frac{7}{n ( \sqrt{1 + \frac{7}{n^2}} + 1)}  \\
		&= 0.
		\end{align*}
	\end{freeResponse}
	
	
	
	\item  $a_n = \frac{(2n+3)!}{5n^3 (2n)!}$
	\begin{freeResponse}
		\begin{align*}
		\lim_{n \to \infty} \frac{(2n+3)!}{5n^3(2n)!}
		&= \lim_{n \to \infty} \frac{(2n+3)(2n+2)(2n+1)(2n)!}{5n^3(2n)!}  \\
		&= \lim_{n \to \infty} \frac{(2n+3)(2n+2)(2n+1)}{5n^3}  \\
		&= \frac{8}{5} 	\quad	{\color{red}\text{Compare the coefficients of the leading }n^3 \text{terms}}
		\end{align*}
	\end{freeResponse}
	
	
	
<<<<<<< HEAD
	\item  $a_n = (2^n + 3^n)^{\frac{1}{n}}$  
	\begin{center}
	{\it Hint:  $a_n \geq (0+3^n)^{\frac{1}{n}} = 3$ and $a_n \leq (2 \cdot 3^n)^{\frac{1}{n}} = 2^{\frac{1}{n}} \cdot 3$}
	\end{center}
=======
	\item  $a_n = (2^n + 3^n)^{\frac{1}{n}}$  \\
	Hint:  $a_n \geq (0+3^n)^{\frac{1}{n}} = 3$ and \\ $a_n \leq (2 \cdot 3^n)^{\frac{1}{n}} = 2^{\frac{1}{n}} \cdot 3$
%	\begin{center}
%	{\it 
%	\end{center}
>>>>>>> origin/master
	\begin{freeResponse}
	From the hint
		\[
		3 = (0+3^n)^\frac{1}{n}  \leq a_n \leq (2 \cdot 3^n)^\frac{1}{n} = 2^\frac{1}{n} \cdot 3.
		\]
	So by the squeeze theorem, we have that
		\begin{align*}
		\lim_{n \to \infty} 3 \leq &\lim_{n \to \infty} a_n \leq \lim_{n \to \infty} 2^\frac{1}{n} \cdot 3  \\
		\Longrightarrow 	\qquad	3 \leq &\lim_{n \to \infty} a_n \leq 1 \cdot 3 = 3.
		\end{align*}
	Thus,
		\[
		\lim_{n \to \infty} a_n = 3.
		\]
	\end{freeResponse}
	
	
	
	\item  $a_n = \frac{n^{365} + 5^n}{8^n + n^3}$
	\begin{freeResponse}
		\begin{align*}
		\lim_{n \to \infty} \frac{n^{365} + 5^n}{8^n + n^3}
		&= \lim_{n \to \infty} \frac{n^{365} + 5^n}{8^n + n^3} \cdot \frac{\frac{1}{8^n}}{\frac{1}{8^n}}  \\
		&=  \lim_{n \to \infty} \frac{\frac{n^{365}}{8^n} + \left( \frac{5}{8} \right)^n}{1 + \frac{n^3}{8^n}}  \\
		&= \frac{0+0}{1+0} = 0.		\quad	{\color{red}\text{due to growth rates,} \lim_{n \to \infty} \frac{n^k}{a^n} = 0.}
		\end{align*}
	\end{freeResponse}
	
	

<<<<<<< HEAD
	
=======
	%\end{multicols}
>>>>>>> origin/master
	\end{enumerate}
	
\end{problem}

\begin{instructorNotes}
For this problem, give each group two of the problems to report on.  
Give about $8$ minutes (or less?) for the group work and about $10-15$ minutes for discussion.  
On (a), L'Hospital's Rule is involved.  
On (e), the squeeze theorem should be involved.  
On (f), the students may use ``known'' facts of the relative growth of polynomial vs. exponential terms.
\end{instructorNotes}







<<<<<<< HEAD
%problem 2
\begin{problem}
Show that 
$$\lim_{n \to \infty} \left( \sqrt{n+1} - \sqrt{n} \right)$$ 
exists by proving that $a_n = \sqrt{n+1} - \sqrt{n}$ is a bounded monotonic sequence.  A hint is to show that $f(x) = \sqrt{x+1} - \sqrt{x}$ is a decreasing function by showing that $f'(x) < 0$.  
	\begin{freeResponse}
	Let $f(x) = \sqrt{x+1} - \sqrt{x}$.  Then
		\begin{align*}
		f'(x) 
		&= \frac{1}{2 \sqrt{x+1}} - \frac{1}{2 \sqrt{x}}  \\
		&= \frac{\sqrt{x} - \sqrt{x+1}}{2 \sqrt{x}\sqrt{x+1}}  \\
		&< 0
		\end{align*}
	since the denominator is clearly positive, and $\sqrt{x} < \sqrt{x+1}$.
	Therefore $f$ is decreasing, and so the original sequence is decreasing.  
	Also notice that since 
	$$\sqrt{x} < \sqrt{x+1}$$
	we have that 
	$$0 < \sqrt{x+1} - \sqrt{x} = f(x).$$
	Thus the original sequence is bounded below by $0$.  \

	Therefore, since the sequence $\left\{ \sqrt{n+1} - \sqrt{n} \right\}$ is bounded and monotone decreasing, the limit
		\[
		\lim_{n \to \infty} \sqrt{n+1} - \sqrt{n}
		\]
	exists.
	\end{freeResponse}
		
\end{problem}

\begin{instructorNotes}
Perhaps do as a whole class discussion.  
Emphasize careful writing of reasoning.
\end{instructorNotes}







%problem 3
\begin{problem}
Find the limit of the given sequence.  
Also, determine if it is a geometric sequence.
	\begin{multicols}{3}
	\begin{enumerate}
	
	\item  $a_n = \frac{n^2}{2^n}$
	%\begin{freeResponse}
	
	%\end{freeResponse}
	
	
	
	\item  $a_n = \frac{1}{3^n}$
	%\begin{freeResponse}
	
	%\end{freeResponse}
	
	
	
	\item  $a_n = \left( \frac{1}{n} \right)^4$
	%\begin{freeResponse}
	
	%\end{freeResponse}
	
	
	
	\item  $a_n = \frac{e^n + (-3)^n}{5^n}$
	%\begin{freeResponse}
	
	%\end{freeResponse}
	
	
	
	\item  $a_n = 3^{\frac{1}{n}}$
	%\begin{freeResponse}
	
	%\end{freeResponse}
	
	\end{enumerate}
	\end{multicols}
	
	\begin{freeResponse}
	\begin{enumerate}
	\item 	$\lim_{n \to \infty} \frac{n^2}{2^n} = 0 	\quad	{\color{red}\text{growth rate}}$
	
	\item  $\lim_{n \to \infty} \frac{1}{3^n} = \lim_{n \to \infty} \left( \frac{1}{3} \right)^n = 0.$
	This is a geometric sequence with $a = 1$ and $r = \frac{1}{3}$.  
	
	\item  $\lim_{n \to \infty} \left( \frac{1}{n} \right)^4 = 0$.  
	
	\item  $\lim_{n \to \infty} \frac{e^n + (-3)^n}{5^n} = \lim_{n \to \infty} \left[ \left( \frac{e}{5} \right)^n + \left( \frac{-3}{5} \right)^n \right] = 0.$  
	
	This is the sum of two geometric sequences.  
	For both, the initial term is $a = 1$.  
	For the first sequence the ratio is $r_1 = \frac{e}{5}$, and for the second the ratio is $r_2 = \frac{-3}{5}$.
	
	\item  $\lim_{n \to \infty} 3^\frac{1}{n} = 3^0 = 1$.  
	
	\end{enumerate}
	\end{freeResponse}

\end{problem}

\begin{instructorNotes}
These limits should be relatively easy to analyze.  
The students need to identify the ``$r$'' if it is a geometric sequence (and note that the exponent $n$ is the variable).  
On (d) and (f), they should argue that they are looking at the sum of two geometric sequences.  
Maybe give one per group with about $8$ minutes for discussion.  
\end{instructorNotes}







=======
>>>>>>> origin/master
%\begin{problem}
%Determine if the following series converge or diverge.  If they converge, find the sum.
%	\begin{enumerate}
%	
%	\item  $e + 1 + e^{-1} + e^{-2} + e^{-3} + \hdots$
%	\begin{freeResponse}
%		\begin{align*}
%		e + 1 + e^{-1} + e^{-2} + e^{-3} + \hdots
%		&= e + \sum_{k = 0}^\infty e^{-k}  \\
%		&= e + \sum_{k=0}^\infty \left( e^{-1} \right)^k 	\quad	{\color{red}\text{geometric series, }r = e^{-1} < 1}  \\
%		&= e + \frac{1}{1-e^{-1}}.
%		\end{align*}
%	Therefore, this series converges to $\left( e + \frac{1}{1-e^{-1}} \right)$.  
%	\end{freeResponse}
%	
%	
%	
%	\item  $\sum_{k=0}^{99} 2^k + \sum_{k=100}^\infty \frac{1}{2^k}$
%	\begin{freeResponse}
%	Let us analyze the two different summands in this problem:
%		\begin{enumerate}
%		\item[(i)]  $\sum_{k=0}^{99} 2^k$
%		
%		This is a finite sum from a geometric sequence, and so its sum is 
%			\[
%			\frac{a(1-r^n)}{1-r}.
%			\]
%		Thus,
%	  		\[
%	  		\sum_{k=0}^{99} 2^k = \frac{1(1-2^{100})}{1-2} = 2^{100} - 1.
%	  		\]
%	  		
%		\item[(ii)]  $\sum_{k=100}^\infty \frac{1}{2^k} = \sum_{k=100}^\infty \left( \frac{1}{2} \right)^k$.  
%		
%		This is a geometric series with $a=\frac{1}{2^{100}}$ and $r = \frac{1}{2}$.  
%		So
%			\[
%			\sum_{k=100}^\infty \frac{1}{2^k} = \frac{\frac{1}{2^{100}}}{1-\frac{1}{2}} = \frac{1}{2^{99}}.
%			\]
%			
%	Therefore, combining parts (i) and (ii) we have that
%		\[
%		\sum_{k=0}^{99} 2^k + \sum_{k=100}^\infty \frac{1}{2^k} = 2^{100} - 1 + \frac{1}{2^{99}}.
%		\]
%		\end{enumerate}
%	\end{freeResponse}
%	
%	
%	
%	\item  $\sum_{k=0}^\infty (\cos(1))^k$
%	\begin{freeResponse}
%	This is a geometric series with $a=1$ and $r = \cos(1)$.  
%	We know that $-1 < \cos(1) < 1$, and so $|\cos(1)|<1$.  
%	Therefore, this geometric series converges and
%		\begin{align*}
%		\sum_{k=0}^\infty (\cos(1))^k = \frac{1}{1-\cos(1)}.
%		\end{align*}
%	\end{freeResponse}
%	
%	
%	
%	\item  $\sum_{k=4}^\infty \frac{5 \cdot 4^{k+3}}{7^{k-2}}$
%	\begin{freeResponse}
%	Let us first reindex this series.  
%	Let $\ell = k-4$.  
%	Then $k=\ell+4$, and when $k=4$, $\ell = 0$.
%	We then have that
%		\begin{align*}
%		\sum_{k=4}^\infty \frac{5 \cdot 4^{k+3}}{7^{k-2}}
%		&= \sum_{\ell=0}^\infty \frac{5 \cdot 4^{\ell + 4 + 3}}{7^{\ell + 4 - 2}}  \\
%		&= \sum_{\ell=0}^\infty \frac{5 \cdot 4^{\ell + 7}}{7^{\ell + 2}}  \\
%		&= \sum_{\ell=0}^\infty \frac{5 \cdot 4^7 \cdot 4^\ell}{7^2 \cdot 7^\ell}  \\
%		&= \frac{5 \cdot 4^7}{7^2} \sum_{\ell=0}^\infty \left( \frac{4}{7} \right)^\ell  \quad	{\color{red}\text{assuming this series converges}}  \\
%		&= \frac{5 \cdot 4^7}{7^2} \cdot \frac{1}{1-\frac{4}{7}}  \quad  {\color{red}\text{geometric series with }a=1, r = \frac{4}{7}}  \\
%		&= \frac{5 \cdot 4^7}{3 \cdot 7}.
%		\end{align*}
%	Therefore, this series converges to $\frac{5 \cdot 4^7}{3 \cdot 7}$.  
%	\end{freeResponse}
%	
%	
%	
%	\item  $\sum_{k=0}^\infty e^{5-2k}$
%	\begin{freeResponse}
%		\begin{align*}
%		\sum_{k=0}^\infty e^{5-2k}
%		&= \sum_{k=0}^\infty \left[ e^5 \cdot \left( e^{-2} \right)^k \right]  \\
%		&= e^5 \sum_{k=0}^\infty \left( e^{-2} \right)^k  	\quad	{\color{red}\text{assuming the series converges}}  \\
%		&= e^5 \cdot \frac{1}{1-e^{-2}}  \quad 	{\color{red}\text{geometric series with }a=1, r=e^{-2} < 1}
%		\end{align*}
%	Therefore, this series converges to $\frac{e^5}{1-e^{-2}}$.  
%	\end{freeResponse}
%	
%	
%	
%	\item  $\sum_{k=0}^\infty \frac{e^k + (-7)^k}{5^k}$
%	\begin{freeResponse}
%		\begin{align*}
%		\sum_{k=0}^\infty \frac{e^k + (-7)^k}{5^k} 
%		&= \sum_{k=0}^\infty \left[ \frac{e^k}{5^k} + \frac{(-7)^k}{5^k} \right]  \\
%		&= \sum_{k=0}^\infty \left[ \left( \frac{e}{5} \right)^k + \left( \frac{-7}{5} \right)^k \right] . \\
%		\end{align*}
%	If both of these series were convergent, then we would be able to split up the sum:
%		\[  
%		\sum_{k=0}^\infty \frac{e^k + (-7)^k}{5^k} ``=" \sum_{k=0}^\infty \left( \frac{e}{5} \right)^k + \sum_{k=0}^\infty \left( \frac{-7}{5} \right)^k.
%		\]
%	The first series on the right hand side is a geometric series with $r=\frac{e}{5}$.  
%	Since $\biggr| \frac{e}{5} \biggr| < 1$, this series converges.  
%	But the second series is a geometric series with $r = \frac{-7}{5}$.  
%	Since $\biggr| \frac{-7}{5} \biggr| > 1$, this series diverges.
%	
%	Therefore, the original series diverges.
%	\end{freeResponse}
%	
%	
%	
%	\item  $ \sum_{k=0}^\infty \left[ \frac{5}{(k+1)(k+2)} + \left( - \frac{1}{2} \right)^k \right]$
%	\begin{freeResponse}
%	If both series converge, then we can break up the sum:
%		\[
%		\sum_{k=0}^\infty \left[ \frac{5}{(k+1)(k+2)} + \left( - \frac{1}{2} \right)^k \right] = \sum_{k=0}^\infty \frac{5}{(k+1)(k+2)} + \sum_{k=0}^\infty \left( - \frac{1}{2} \right)^k.
%		\]
%	Let us consider both series on the right hand side of this equation individually.
%		\begin{enumerate}
%		\item[(i)]  $\sum_{k=0}^\infty \left( - \frac{1}{2} \right)^k$
%		
%		This is a geometric series with $a=1$ and $r = \frac{-1}{2}$.  
%		Therefore, this series converges with
%			\[
%			\sum_{k=0}^\infty \left( - \frac{1}{2} \right)^k = \frac{1}{1- \left( \frac{-1}{2} \right)}  = \frac{2}{3}.
%			\]
%		
%		\item[(ii)]  $\sum_{k=0}^\infty \frac{5}{(k+1)(k+2)}$
%		
%		It may not be obvious yet, but this is a telescoping series.  
%	To see this, let us decompose $\frac{5}{(k+1)(k+2)}$ as a partial fraction.
%		\begin{align*}
%		&\frac{5}{(k+1)(k+2)} = \frac{A}{k+1} + \frac{B}{k+2}  \\
%		\Longrightarrow 	\qquad 	&5 = A(k+2) + B(k+1).
%		\end{align*}
%	We solve for $A$ and $B$ by choosing ``smart" values for $k$:
%		\begin{align*}
%		&(k=-1) 	\quad	\Longrightarrow 	\quad	A = 5  \\
%		&(k=-2) 	\quad	\Longrightarrow		\quad	-B = 5 	\quad	\Longrightarrow		\quad	B = -5.
%		\end{align*}
%	So we see that
%		\begin{align*}
%		\sum_{i=1}^\infty \left( \frac{1}{i} - \frac{1}{i+2} \right) = \sum_{k=0}^\infty \left[ \frac{5}{k+1} - \frac{5}{k+2} \right].
%		\end{align*}
%	Let 
%		\[
%		S_n = \sum_{k=0}^n \left[ \frac{5}{k+1} - \frac{5}{k+2} \right].
%		\]
%	Then we have that
%		\begin{align*}
%		S_n &= \sum_{k=0}^n \left[ \frac{5}{k+1} - \frac{5}{k+2} \right]  \\
%		&= \left( \frac{5}{1} - \frac{5}{2} \right) + \left( \frac{5}{2} - \frac{5}{3} \right) + \left( \frac{5}{3} - \frac{5}{4} \right) + \hdots + \left( \frac{5}{n+1} - \frac{5}{n+2} \right)  \\
%		&= \frac{5}{1} - \frac{5}{n+2} = 5 - \frac{5}{n+2}.
%		\end{align*}
%	We then compute the sum by taking the limit of the sequence of partial sums:
%		\begin{align*}
%		\sum_{k=0}^\infty \frac{5}{(k+1)(k+2)}
%		&= \lim_{n \to \infty} \sum_{k=0}^n \frac{5}{(k+1)(k+2)}  \\
%		&= \lim_{n \to \infty} S_n  \\
%		&= \lim_{n \to \infty} \left( 5 - \frac{5}{n+2} \right)  \\
%		&= 5.
%		\end{align*}
%		\end{enumerate}
%	\vskip 5pt
%	Finally, we compute the sum of the original series as
%		\begin{align*}
%		\sum_{k=0}^\infty \left[ \frac{5}{(k+1)(k+2)} + \left( - \frac{1}{2} \right)^k \right] 
%		&= \sum_{k=0}^\infty \frac{5}{(k+1)(k+2)} + \sum_{k=0}^\infty \left( - \frac{1}{2} \right)^k  \\
%		&= 5 + \frac{2}{3} = \frac{17}{3}.
%		\end{align*}
%	\end{freeResponse}
%	
%	
%	
%	\item  $\sum_{i=1}^\infty \left( \frac{1}{i} - \frac{1}{i+2} \right)$
%	
%	\begin{freeResponse}
%	This is a telescoping series.  
%	Let 
%		\[
%		S_n = \sum_{i=1}^n \left( \frac{1}{i} - \frac{1}{i+2} \right).
%		\]
%	Then,
%		\begin{align*}
%		S_n &= \sum_{i=1}^n \left( \frac{1}{i} - \frac{1}{i+2} \right)  \\
%		&= \left( {\color{red}\frac{1}{1}} - \frac{1}{3} \right) + \left( {\color{red}\frac{1}{2}} - \frac{1}{4} \right) + \left( \frac{1}{3} - \frac{1}{5} \right) + \left( \frac{1}{4} - \frac{1}{6} \right) + \left( \frac{1}{5} - \frac{1}{7} \right)  \\
%		&+ \hdots + \left( \frac{1}{n-2} - \frac{1}{n} \right) + \left( \frac{1}{n-1} - {\color{red}\frac{1}{n+1}} \right) + \left( \frac{1}{n} - {\color{red}\frac{1}{n+2}} \right)  \\
%		&= 1 + \frac{1}{2} - \frac{1}{n+1} - \frac{1}{n+2}.
%		\end{align*}
%	Note that the last equality above is because all of the non-red terms cancel (convince yourself of this).  
%	Then
%		\begin{align*}
%		\sum_{i=1}^\infty \left( \frac{1}{i} - \frac{1}{i+2} \right)
%		&= \lim_{n \to \infty} S_n  \\
%		&= \lim_{n \to \infty} \left[ 1 + \frac{1}{2} - \frac{1}{n+1} - \frac{1}{n+2} \right]  \\
%		&= 1 + \frac{1}{2} = \frac{3}{2}.
%		\end{align*}
%	\end{freeResponse}
%	
%	\end{enumerate}
%	
%\end{problem}
%
%\begin{instructorNotes}
%Assign two per group.  
%Most of these involve geometric series and one (part (b)) involves a finite geometric sum, whose ``trick'' is presented in the lecture.  
%
%Students must identify the ``$r$'' and pay attention to both the indices and the exponents (for example, $7^{k+3} = 7^3 \cdot 7^k$).  
%Encourage multiple methods (there are $3$ methods presented in the lecture) and make sure students clearly explain their reasoning on paper.  
%\end{instructorNotes}
%
%
%
%
%
%
%
%%problem 2
%\begin{problem}
%Convert the decimal $2.456\overline{314}$ to a fraction using geometric series.
%	\begin{freeResponse}
%		\begin{align*}
%		2.456\overline{314}
%		&= 2.456 + 0.000314 + 0.000000314 + \hdots  \\
%		&= 2.456 + \frac{314}{1000^2} + \frac{314}{1000^3} + \hdots  \\
%		&= 2.456 + \sum_{k=1}^\infty \left[ \frac{314}{1000} \cdot \left( \frac{1}{1000} \right)^k \right]  \\
%		&= \frac{2456}{1000} + \frac{\frac{314}{1000^2}}{1 - \frac{1}{1000}}  \\
%		&=  \frac{2456}{1000} + \frac{\frac{314}{1000^2}}{\frac{999}{1000}}  \\
%		&= \frac{2456}{1000} + \frac{314}{999000}  \\
%		&= \frac{2453544 + 314}{999000}  \\
%		&= \frac{2453858}{999000} = \frac{1226929}{499500}.
%		\end{align*}
%	\end{freeResponse}
%
%\end{problem}
%
%\begin{instructorNotes}
%This is a common type of problem in this section.  
%Students have (most likely) never seen a problem with a non-repeating part to the decimal.  
%This should probably be done as a whole class.
%\end{instructorNotes}
%
%
%
%
%
%
%
%
%%problem 3
%\begin{problem}
%Find all values of $x$ for which the series 
%$$f(x) = \sum_{k=0}^\infty \frac{(x+3)^k}{2^k}$$ 
%converges.
%	\begin{freeResponse}
%	First notice that
%		\[
%		f(x) = \sum_{k=0}^\infty \frac{(x+3)^k}{2^k} = \sum_{k=0}^\infty \left( \frac{x+3}{2} \right)^k
%		\]
%	and so this is a geometric series with $a=1$ and $r = \frac{x+3}{2}$.  
%	So this series converges when
%		\begin{align*}
%		\biggr| &\frac{x+3}{2} \biggr| < 1  \\
%		\Longleftrightarrow 	\quad	-1 < &\frac{x+3}{2} < 1  \\
%		\Longleftrightarrow		\quad	-2 < &x+3 < 2  \\
%		\Longleftrightarrow		\quad	-5 < &x < -1.
%		\end{align*}
%	\end{freeResponse}
%		
%\end{problem}
%
%\begin{instructorNotes}
%The purpose of this problem is to give the students a preview of the idea of an interval of convergence (to be covered in Chapter 10).  
%Students need to be careful with absolute values and behavior at endpoints.  
%This could be skipped (or presented as a ``take home and think about it'' question).  
%\end{instructorNotes}
%
%
%
%
%
<<<<<<< HEAD




=======
%
%
%
%
>>>>>>> origin/master


	
	
	
	
	
	
	
	
	

	










								
				
				
	














\end{document} 


















