\documentclass[noinstructornotes]{ximera}
%handout:  for handout version with no solutions or instructor notes
%handout,instructornotes:  for instructor version with just problems and notes, no solutions
%noinstructornotes:  shows only problem and solutions

%% handout
%% space
%% newpage
%% numbers
%% nooutcomes

%I added the commands here so that I would't have to keep looking them up
%\newcommand{\RR}{\mathbb R}
%\renewcommand{\d}{\,d}
%\newcommand{\dd}[2][]{\frac{d #1}{d #2}}
%\renewcommand{\l}{\ell}
%\newcommand{\ddx}{\frac{d}{dx}}
%\everymath{\displaystyle}
%\newcommand{\dfn}{\textbf}
%\newcommand{\eval}[1]{\bigg[ #1 \bigg]}

%\begin{image}
%\includegraphics[trim= 170 420 250 180]{Figure1.pdf}
%\end{image}

%add a ``.'' below when used in a specific directory.
\input{../preamble.tex} %% we can turn off input when making a master document

\title{Recitation \# 5:  Length of Curves \& Surface Area}  

\begin{document}
\begin{abstract}		\end{abstract}
\maketitle



\begin{comment}
\section{Warm up:}

	\begin{freeResponse}
	
	\end{freeResponse}
	
\begin{instructorNotes}

\end{instructorNotes}
\end{comment}







\section{Group work:}



%problem 1





%problem 1
\begin{problem}
Find the length of the following curves (length is in feet):
	\begin{enumerate}
		\item  $y = \frac{4}{3} x^\frac{3}{2}$ from $\left( 0, 0 \right)$ to $\left( 1, \frac{4}{3} \right)$.  
		\begin{freeResponse}
			\begin{align*}
			\text{{\color{red} Arc Length}} &= \int_0^1 \sqrt{1+y'(x)^2} \d x  \\
			&= \int_0^1 \sqrt{1+ \left( 2x^{\frac{1}{2}}\right)^2} \d x  \\
			&=  \int_0^1 \sqrt{1+ 4x} \d x  \\
			& \\
			u &= 1+4x\\
			du &= 4dx \\
			\frac{du}{4} &= dx \\
			& \\
			u(0) &= 1 + 4(0) =1\\
			u(1) &= 1 +4(1) = 5\\
			& \\
			&= \frac{1}{4} \int_1^5 \sqrt{u} \d u  \\
			&= \eval{\frac{2}{3}x^{\frac{3}{2}}}_1^5  \\
			&= \left( \frac{2}{3}(5)^{\frac{3}{2}} \right) - \left( \frac{2}{3}(1)^{\frac{3}{2}} \right)  
			=\frac{2}{3}(5)^{\frac{3}{2}}  -  \frac{2}{3} 
			\end{align*}
		\end{freeResponse}
		
		
		
		\item  $x = \frac{1}{9} e^{3y} + \frac{1}{4} e^{-3y}$ from $\left( \frac{13}{36}, 0 \right)$ to $\left( \frac{265}{288}, \ln 2 \right)$.  
		\begin{freeResponse}
			\begin{align*}
			\text{{\color{red} Arc Length}} &= \int_0^{\ln 2} \sqrt{1+x'(y)^2} \d y  \\
			&=\int_0^{\ln 2} \sqrt{1+ \left( \frac{1}{3}e^{3y} - \frac{3}{4}e^{-3y} \right)^2} \d y  \\
			&=  \int_0^{\ln 2} \sqrt{1+ \left( \frac{1}{9}e^{6y} - \frac{1}{2} + \frac{9}{16}e^{-6y} \right)} \d y  \\
			&= \int_0^{\ln 2} \sqrt{ \frac{1}{9}e^{6y} + \frac{1}{2} + \frac{9}{16}e^{-6y} } \d y  \\
			&= \int_0^{\ln 2} \sqrt{ \left( \frac{1}{3}e^{3y} + \frac{3}{4}e^{-3y} \right)^2} \d y  \\
			&= \int_0^{\ln 2} \left( \frac{1}{3}e^{3y} + \frac{3}{4}e^{-3y} \right) \d y  \\
			&= \eval{ \frac{1}{9}e^{3y} - \frac{1}{4}e^{-3y}}_0^{\ln 2}  \\
			&\overset{*}{=} \left( \frac{8}{9} - \frac{1}{32} \right) - \left( \frac{1}{9} - \frac{1}{4} \right)  \\
			&= \frac{7}{9} + \frac{7}{32} = \frac{224 + 63}{288} = \frac{287}{288}.
			\end{align*}
		* Note that
			\[
			e^{3 \ln 2} = e^{\ln 2^3} = e^{\ln 8} = 8
			\]
	and
			\[
			e^{-3 \ln 2} = e^{\ln 2^{-3}} = 2^{-3} = \frac{1}{8}.
			\]
		\end{freeResponse}
		
	\end{enumerate}

\end{problem}

\begin{instructorNotes}
Split (a) and (b) amont the groups.  
Note that the focus here is on both the set-up \dfn{and} in solving the resulting integral.
\end{instructorNotes}



%problem 2
\begin{problem}
Find the surface area of the surface generated by revolving the curve given by
	\begin{enumerate}
			\item  $x = 2y^3$ from $\left( 0, 0 \right)$ to $\left( 2, 1 \right)$ about the $y$-axis.
		\begin{freeResponse}
		The formula for the surface area is
			\[
			\text{{\color{red} Surface Area}} = \int_0^{1} 2 \pi f(y) \sqrt{1+f'(y)^2} \d y.
			\]
		Since $x = f(y) = 2y^3$, 
		we know that $f'(y) = 6y^2$.  
		Note that
			\begin{align*}
			\sqrt{1+f'(y)^2} \d y  &= \sqrt{1+ \left( 6y^2 \right)^2}  \\
			&=  \sqrt{1+ 36y^4}  \\
			\end{align*}
		and so
			\begin{align*}
			\text{{\color{red} Surface Area}} &= \int_0^{1} 2 \pi \left(2y^3\right) \left(  \sqrt{1+ 36y^4} \right) \d y  \\
			&= \int_0^{1} 4 \pi y^3  \sqrt{1+ 36y^4}  \d y \\
			& \\
			u&=1+ 36y^4 \\
			du &= 144y^3dy \\
			\frac{du}{144} &= y^3dy \\
			& \\
			u(0)&=1+36(0)^4 = 1\\
			u(1)&= 1+ 36(1)^4) = 37 \\
			& \\
			&= \frac{4 \pi}{144} \int_0^{37}  \sqrt{u}  \d u \\
			&= \frac{4 \pi}{144} \eval{\frac{2}{3}u^{\frac{3}{2}}}_0^{37}\\
			&= \frac{4 \pi}{144} \left[ \left( \frac{2}{3}(37)^{\frac{3}{2}} \right) - \left( \frac{2}{3}(0)^{\frac{3}{2}} \right) \right]  \\
			&= \frac{8(37)^{\frac{3}{2}} }{432}\pi
			\end{align*}
	
		\end{freeResponse}
	
	
		\item  $y = \frac{1}{6} x^3 + \frac{1}{2x}$ from $\left( 2, \frac{19}{12} \right)$ to $\left( 3, \frac{14}{3} \right)$ about the $x$-axis.
		\begin{freeResponse}
		The formula for the surface area is
			\[
			\text{{\color{red} Surface Area}} = \int_2^3 2 \pi f(x) \sqrt{1+f'(x)^2} \d x.
			\]
		Since $y = f(x) = \frac{1}{6} x^3 + \frac{1}{2x}$, we know that $f'(x) = \frac{1}{2} x^2 - \frac{1}{2} x^{-2}$.  
		Note that
			\begin{align*}
			\sqrt{1+f'(x)^2} &= \sqrt{1+ \left( \frac{1}{2}x^2 - \frac{1}{2}x^{-2} \right)^2}  \\
			&= \sqrt{1+ \left( \frac{1}{4}x^4 - \frac{1}{2} + \frac{1}{4}x^{-4} \right)}  \\
			&= \sqrt{\frac{1}{4}x^4 + \frac{1}{2} + \frac{1}{4}x^{-4}}  \\
			&= \sqrt{\left( \frac{1}{2}x^2 + \frac{1}{2}x^{-2} \right)^2}  \\
			&= \left( \frac{1}{2}x^2 + \frac{1}{2}x^{-2} \right)
			\end{align*}
		and so
			\begin{align*}
			\text{{\color{red} Surface Area}} &= \int_2^3 2 \pi \left( \frac{1}{6} x^3 + \frac{1}{2} x^{-1} \right) \left( \frac{1}{2} x^2 + \frac{1}{2} x^{-2} \right) \d x  \\
			&= 2 \pi \int_2^3 \left( \frac{1}{12} x^5 + \frac{1}{12} x + \frac{1}{4} x + \frac{1}{4} x^{-3} \right) \d x  \\
			&= 2 \pi \int_2^3 \left( \frac{1}{12} x^5 + \frac{1}{3} x + \frac{1}{4} x^{-3} \right) \d x  \\
			&= 2 \pi \eval{\frac{1}{72}x^6 + \frac{1}{6}x^2 - \frac{1}{8}x^{-2}}_2^3  \\
			&= 2\pi \left[ \left( \frac{81}{8} + \frac{3}{2} - \frac{1}{72} \right) - \left( \frac{8}{9} + \frac{2}{3} - \frac{1}{32} \right) \right]  \\
			&= 2\pi \left( \frac{2916 + 432 - 4 - 256 - 192 + 9}{288} \right)  \\
			&= \frac{2905 \pi}{144}.
			\end{align*}
		\end{freeResponse}
		
		
		

	\end{enumerate}
	
\end{problem}

\begin{instructorNotes}
Split (a) and (b) among the same groups as before.  
Make sure that the students are using the surface area formula and not the arc length formula.
\end{instructorNotes}



%problem 3
\begin{problem}
{\bf Set up} an integral (or a {\it sum of integrals}) to find the perimeter of the region bounded by the curves $y=2x^2-5x+13$ and $y=x^2+6x-11$.
	\begin{freeResponse}
	Let $f(x) = 2x^2-5x+13$ and $g(x) = x^2+6x-11$.
	We first need to find the points where these two curves intersect.  
	So we solve
		\begin{align*}
		f(x) &= g(x)  \\
		2x^2-5x+13 &= x^2+6x-11  \\
		x^2-11x+24 &= 0  \\
		(x-3)(x-8) &= 0  \\
		x &= 3,8.
		\end{align*}
	Then the perimeter is $L_1 + L_2$ where
		\begin{align*}
		L_1 &= \int_3^8 \sqrt{1+f'(x)^2} \d x = \int_3^8 \sqrt{1+(4x-5)^2} \d x  \\
		L_2 &= \int_3^8 \sqrt{1+g'(x)^2} \d x = \int_3^8 \sqrt{1+(2x+6)^2} \d x.
		\end{align*}	
	\end{freeResponse}
	
\end{problem}

\begin{instructorNotes}
It may be helpful to remind students that perimeter makes sense for more general two-dimensional shapes.
\end{instructorNotes}




%problem 4
\begin{problem}
A steady wind blows a kite due west.  
The kite's height above the ground from horizontal position $x=0$ ft. to $x=80$ ft. is given by
	\[
	y = 150 - \frac{1}{40}(x-50)^2.
	\]
Set up the integral to find the distance traveled by the kite.  
	\begin{freeResponse}
		\[
		\text{{\color{red} distance the kite traveled}} = \int_0^{80} \sqrt{1+y'(x)^2} \d x
		\]
	where $y(x) = 150 - \frac{1}{40}(x-50)^2$.  
	Then since $y' = - \frac{1}{20}(x-50)$, we have that
		\[
		\text{{\color{red} distance the kite traveled}} = \int_0^{80} \sqrt{1 + \frac{(x-50)^2}{400}} \d x.
		\]
	\end{freeResponse}
		
\end{problem}

\begin{instructorNotes}
After discussing the problem, you might ask what the difference is between the given problem and if, say we are given the height of a ball at any time $t$ and asked to find the distance that the ball traveled.  
In this set-up, the height is not the \dfn{path} of the ball and thus the length of the curve does not represent the distance that the ball traveled.  
\end{instructorNotes}













	
	
	
	
	
	
	
	
	

	










								
				
				
	














\end{document} 


















